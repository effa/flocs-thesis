\chapter{Design of RoboMission}
\label{chap:design-of-robomission}

\begin{itemize}
\item what = app for efficient learning of elementary programming for children
\item entertaining problems of optimal difficulty
\item max \emph{flow} $\rightarrow$ max efficiency
\item linearized description: design (this chapter), implementation (nexth chapter); in reality: many iterations
\end{itemize}


\imgW{robomission-task1}{Task Environment of the RoboMission.}


\section{Goals}
\label{sec:robomission.goals}

\begin{itemize}
\item TBA, with respect to objectives section in previous chapter, but be more specific
\end{itemize}




\section{Use Cases}
\label{sec:robomission.use-cases}

\begin{itemize}
\item single learner mode is the only one already supported
\item Hour of Code  - single hour,  mainly as a motivation and promotion
\item primary school mode (RoboBlocks) -> needs: teacher mode and guide
\item secondary school mode (RoboCode)
\item KSI -- 0th problem set - needs Python
\item IB111 -- 0th/1st motivational lesson - needs Python and to be better than turtle
\item MjUNI workshop
\item ? competitions such as Purkiada (physical version already in InterSoB 2017)
%\item education exhibit at VIDA  Science Center
\item -> functional requirements
\end{itemize}




\section{Game World}
\label{sec:robomission.game-world}

\begin{itemize}
\item original version - robot-in-maze, issues:
  \begin{itemize}
  \item not flexible and did not allow for a lot of diverse easy tasks,
  \item which is crucial for adaptive learning system
  \item unnecessarily long programs for simple ideas (non-elegant programs) (TBA: show comparision)
  \item non-elegant tasks
  \item ugly worlds
  \item not very original and entertaining for kids
  \end{itemize}
\item principle: game itself must be fun to play with (ref: Book of Lenses)
\item new game: rocket in space - different theme, mechanics, graphics
\item world elements: grid with colors (color clouds), final line, asteroids and meteoroids, diamonds, wormholes (TODO: explain why), spaceship
\item spaceship actions (left, fly, right, shoot) and sensors (color, position), limited energy
\item explain (and demonstrate) the usefulness of contnuous flight ahead (shorter programs, eliminate the common turning left-right confusion), but also disadvantages: (1) more difficult to use space effiently, this was partially solved by the wormholes hack, another potential solution: more spaceships), (2) it is not what users expect the first time they see the world and left/right action
\end{itemize}


\section{Tasks}
\label{sec:robomission.tasks}

\begin{itemize}
\item space world as described above, single program (blocky/roboCode), length limit
\item goal: reach the final row and collect all diamonds on the way
\item blocks: depends on the level, actions, sensing, repeat, while, if, if-else
\end{itemize}

\imgW{prototype-pits}{Lesson from the first prototype: world must contain elements that the program cannot check to enable diverse tasks. As we had a natural sensor to check walls, we needed to add pits that could not be checked by any sensor.}


\imgW{robomission-task2}{Task with asteroids.}
\imgW{robomission-task3}{Task with colors and wormholes.}


\section{Task Editor}
\label{sec:robomission.system}

\begin{itemize}
\item it's much easier to create new tasks then in first prototype
\end{itemize}


\imgW{task-editor}{Task editor allows to create new tasks.}
\imgW{task-editor-vim}{Task editor allows to to write solutions in Python-based RoboCode and edit space world in Vim mode.}


\section{Learning System}
\label{sec:robomission.system}

\begin{itemize}
\item components: home, task environment, task completion dialog and recommendations, statistics (tasks overview), (feedback form), (login) (task editor)
\item intuitive and simple user interface crucial (aiming at children, they need to focus on learning programming, it would be bad to waste their mental power on understanding a complex interface)
\item mini-instructions (ref to the Google research on ignoring instructions, show how it was solved in Blockly Games; ref figure)
\item mini-explations (difference to instructions: after the fact) (ref figure) (they also serve as a convenient mean to game resetting)
\item motivation: intrinsic (fun challenging game + optimal difficulty) and simple external motivation scheme: credits and levels
\item levels -- currently 9, how many block available (usually one new concept in each level)
\end{itemize}


\imgW{robomission-mini-instruction}{When a student encounters a new programming concept or programming element, the system displays a short instruction.}
\imgW{robomission-mini-explanation}{The system shows an explanation why the exection was unsuccessful.}
\imgW{robomission-levels-credits}{Students earn credits after each solved task.}
\imgW{robomission-tasks-overview}{Overview of all tasks. There is currently 9 levels, each containing about 8 tasks.}

\imgW{prototype-instructions}{Instructions in the first prototype. Nobody was reading them. Most people even did not notice there are any instructions.}


\section{Monitoring}
\label{sec:robomission.monitoring}

\begin{itemize}
\item crucial component for iterative improvement (see Rule of the Loop section below)
\end{itemize}


\section{Non-functional Requirements}
\label{sec:robomission.nonfunctional-requirements}

\begin{itemize}
\item TBA
\end{itemize}



\section{Rule of the Loop}
\label{sec:robomission.rule-of-the-loop}

\begin{itemize}
\item the design interleaved with the implementation -> final design is completely different than the original one
\item ref: rule of the loop (ref: in game desing: Book of Lenses, for ML: Google Rules of ML, for WS development: SCRUM; our project includes all these 3 elements)
\item first prototype: 2016, one year of development, thrown away; for testing our initial ideas and find what works and what not; 50 tasks with a robot in maze
  \begin{itemize}
  \item problems with the robot in the maze: not fun (boring, not inovative, repetitiveness), did not allow for a plenty of diverse easy tasks (which is necessary for adaptive systems), required a lot of blocks even for simple programs (compared to the SpaceGame)
  \item problems with the codebase (maintainability, extensibility - why?)
  \item and the good things? (SPA, explored/verified useful technologies, such as Blockly and Django)
  \end{itemize}
\end{itemize}

\imgW{prototype-task-environment}{First prototype of the system, with a classic robot-in-maze game.}
