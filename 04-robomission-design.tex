\chapter{Design of RoboMission}
\label{chap:design-of-robomission}

\begin{itemize}
\item RoboMission = app for efficient and fun learning of elementary programming for children (= mission)
\item how: entertaining problems of optimal difficulty $\rightarrow$ max \emph{flow} $\rightarrow$ max fun, learning efficiency, happiness
\item "robot in grid world" and Blockly blocks
\item theme: spaceship in space, collecting diamonds
\item covered concepts: sequences of commands, repeat, while, if, if-else, simple tests (comparing)
\item linearized description: design (this chapter), implementation (nexth chapter); in reality: many iterations
\end{itemize}


\imgW{robomission-task1}{Example of a task in the RoboMission app.}


\section{Expected System Behavior}
\label{sec:robomission.behavior}

Student: (anybody who can at least read? targeting at?)
\begin{itemize}
\item All presented tasks are fun, challenging, pleasant to look at, they "call to be solved".
\item Student is able to solve any task recommended by the system (within 15 minutes).
\item Student is able to solve the first few tasks quickly (within 2 minutes).
\item Student is not bored by neither too easy task requiring many command (i.e. taking more than 1 minute to build the solution), nor by the sequence of either too easy or too similar tasks.
\item Skilled student should progress throught first few level quickly (within 10 minutes) and get to the more challenging tasks containing both types of loops, conditionals etc.
\item Each new concept (e.g. block) is explained in the task where it appears for the first time (for given student) (and not earlier).
\item Student understands all elements in the game world, available blocks and in the task environment.
\item TBA: motivation and feel of progress
\item TBA: statistics for student
\end{itemize}

Admin/Researcher:
\begin{itemize}
\item Admin can immediately see how much is the system used and how the system behaves with respect to the short-term ("live-evaluation") metrics (...)
\item Admin receives feedback from provided by users, and error reports on unhandled exceptions.
\item Admin can see metrics on individual tasks (to quickly detect issues with a task).
\item Researcher: TBA - provided data (exports), exploration aspect of recommendation
\end{itemize}




\section{Use Cases}
\label{sec:robomission.use-cases}

\begin{itemize}
\item "Hour of Code" mode
  \begin{itemize}
  \item single hour
  \item mainly as a motivation to programming
  \item using RoboBlocks
  \item directly at elemenatry and high schools, or at home
  \item plus: MjUNI workshop
  \item shorter promotianal version for DODs (?) ("10 minutes of code")
  \item (should be strictly time-limited; certificate at the end)
  \end{itemize}
\item "foundations mode" individual learning of elementary programming (individual at home or in a classroom, from several days to several weeks, depending on the prior skill); natural continuation of the first "Hour of Code" (next levels, with RoboBlocks)
\item "university mode" -- levels with RoboCode/Python, at home / secondary schools, KSI (0th problem set), IB111 (0th/1st motivational lesson - needs Python and to be better than turtle)
\item "competition/texting mode" -- competitions such as Purkiada, Pevnost FI, KSI (advanced problem sets), (physical version already in InterSoB 2017); testing mode for RH interns
\item nice thing is that all these modes can be naturally implemented just as distinct levels, going from the easy tasks using RoboBlocks, through the RoboCode/Python, to the difficult tasks prepared for individual competitions
%\item education exhibit at VIDA  Science Center
\item TBA: -> functional requirements (...)
\end{itemize}




\section{Game World}
\label{sec:robomission.game-world}

\begin{itemize}
\item original version - robot-in-maze, issues:
  \begin{itemize}
  \item not flexible and did not allow for a lot of diverse easy tasks,
  \item which is crucial for adaptive learning system
  \item unnecessarily long programs for simple ideas (non-elegant programs) (TBA: show comparision)
  \item non-elegant tasks
  \item ugly worlds
  \item not very original and entertaining for kids
  \end{itemize}
\item principle: game itself must be fun to play with (ref: Book of Lenses)
\item requirements (on the topic/theme/world):
  \begin{itemize}
  \item entertaining for children
  \item allowing to create plenty of diverse (and entertaining) tasks, inluding very simple ones
  \item programs should be not too verbose
  \item not limiting adaptability (e.g. story requring a fixed sequence of tasks would be a problem)
  \item (not limiting another topics later)
  \end{itemize}
\item new game: rocket in space - different theme, mechanics, graphics
\item world elements: grid with colored fields (color clouds) and objects: asteroids and meteoroids, diamonds, wormholes (TODO: explain why), final line, spaceship
\item spaceship actions (left, fly, right, shoot) and sensors (color, position), limited energy
\item explain (and demonstrate) the usefulness of continuous flight ahead (shorter programs, eliminate the common turning left-right confusion), but also disadvantages: (1) more difficult to use space effiently, this was partially solved by the wormholes hack, another potential solution: more spaceships), (2) it is not what users expect the first time they see the world and left/right action
\end{itemize}


% TODO: use different spaceworld than in Imlementation chapter.
\imgW[0.4]{spaceworld}{Example of Space World.}



\section{Tasks}
\label{sec:robomission.tasks}

\begin{itemize}
\item space world as described above, single program (blocky/roboCode), length limit
\item goal: reach the final row and collect all diamonds on the way
\item blocks: depends on the level, actions, sensing, repeat, while, if, if-else
\end{itemize}

\imgW{prototype-pits}{Lesson from the first prototype: world must contain elements that the program cannot check to enable diverse tasks. As we had a natural sensor to check walls, we needed to add pits that could not be checked by any sensor.}


\imgW{robomission-task2}{Task with asteroids.}


\section{Task Editor}
\label{sec:robomission.system}

\begin{itemize}
\item it's much easier to create new tasks then in first prototype
\end{itemize}


\imgW{task-editor}{Task editor allows to create new tasks.}
\imgW{task-editor-vim}{Task editor allows to to write solutions in Python-based RoboCode and edit space world in Vim mode.}


\section{Learning System}
\label{sec:robomission.system}

\begin{itemize}
\item components: home, task environment, task completion dialog and recommendations, statistics (tasks overview), (feedback form), (login) (task editor)
\item intuitive and simple user interface crucial (aiming at children, they need to focus on learning programming, it would be bad to waste their mental power on understanding a complex interface)
\item mini-instructions (ref to the Google research on ignoring instructions, show how it was solved in Blockly Games; ref figure)
\item mini-explations (difference to instructions: after the fact) (ref figure) (they also serve as a convenient mean to game resetting)
\item motivation: intrinsic (fun challenging game + optimal difficulty) and simple external motivation scheme: credits and levels
\item levels -- currently 9, how many block available (usually one new concept in each level)
\item recommendation -- simple filtering (only consider unsolved tasks on the level $\leq$ student level), simple weighting based on difference between student and task levels (exponential decay), rulette wheel selection
\end{itemize}


\imgW{robomission-mini-instruction}{When a student encounters a new programming concept or programming element, the system displays a short instruction.}
\imgW{robomission-mini-explanation}{The system shows an explanation why the exection was unsuccessful.}
\imgW{robomission-levels-credits}{Students earn credits after each solved task.}
\imgW{robomission-tasks-overview}{Overview of all tasks. There is currently 9 levels, each containing about 8 tasks.}

\imgW{prototype-instructions}{Instructions in the first prototype. Nobody was reading them. Most people even did not notice there are any instructions.}


\section{Monitoring}
\label{sec:robomission.monitoring}

\begin{itemize}
\item crucial component for iterative improvement (see Rule of the Loop section below)
\item components: see dashboard; GA, metrics, feedback, error reports
\item TBA: screenshots of dashboard
\item collect metrics related to long-term objectives (as described in \ref{sec:long-term-objectives})
  \begin{itemize}
  \item "daily active students" = number of users who solved at least 1 task this day
  \item "solving hours" = total time spent on successful attempts
  \item "success ratio" = proportion of successful attempts
  \item number of solved tasks
  \item (TBA: update if it changed)
  \end{itemize}
\item also collect metrics about each task (in order to e.g. detect problematic ones)
  \begin{itemize}
  \item solved count
  \item median time
  \item success ratio
  \end{itemize}
\item all these metrics are computed every night and shown in the monitoring dashboard
\end{itemize}


\section{Non-functional Requirements}
\label{sec:robomission.nonfunctional-requirements}

\begin{itemize}
\item easy to understand code, pleasure to read and write (extend)
\item easy to refactor and add new things (new tasks, levels, recommendation strategies etc.)
\item robust, efficient, interpretable behavior
\end{itemize}



\section{Rule of the Loop}
\label{sec:robomission.rule-of-the-loop}

\begin{itemize}
\item the design interleaved with the implementation -> final design is completely different than the original one
\item ref: rule of the loop (ref: in game desing: Book of Lenses, for ML: Google Rules of ML, for WS development: SCRUM; our project includes all these 3 elements)
\item first prototype: 2016, one year of development, thrown away; for testing our initial ideas and find what works and what not; 50 tasks with a robot in maze
  \begin{itemize}
  \item problems with the robot in the maze: not fun (boring, not inovative, repetitiveness), did not allow for a plenty of diverse easy tasks (which is necessary for adaptive systems), required a lot of blocks even for simple programs (compared to the SpaceGame)
  \item problems with the codebase (maintainability, extensibility - why?)
  \item and the good things? (SPA, explored/verified useful technologies, such as Blockly and Django)
  \end{itemize}
\end{itemize}

\imgW{prototype-task-environment}{First prototype of the system, with a classic robot-in-maze game.}
