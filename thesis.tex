%% Master thesis on Adaptive learning of programming.
\documentclass[
    %printed,
    digital,
    color,
    11pt,
    nocover,
    table,  % coloured tables; (to disable: notable)
    nolof,  % hide List of Figures
    nolot,  % hide List of Tables
    microtype,
    %final,
]{fithesis3}
%% Locales.
\usepackage[resetfonts]{cmap}
\usepackage[T1]{fontenc}
\usepackage[
  main=english,
  english, czech
]{babel}
%% Metadata.
\thesissetup{
    date          = \the\year/\the\month/\the\day,
    university    = mu,
    faculty       = fi,
    type          = mgr,
    author        = Tomáš Effenberger,
    gender        = m,
    advisor       = {doc. Mgr. Radek Pelánek, Ph.D},
    title         = {Adaptive System for Learning Programming},
    TeXtitle      = {Adaptive System for\\Learning Programming},
    keywords      = {adaptive learning, intelligent tutoring system, student modeling, learning programming},
}
%% Bibliography.
\usepackage[backend=biber, 		% use biber as backend instead of BiBTeX
	bibstyle=ieee-alphabetic, 	% bibliography style: IEEE with alphabetic citations
	citestyle=alphabetic, 		% citation style
	url=true, 			        % display urls in bibliography
	hyperref=auto,			    % detect hyperref and create links
]{biblatex}
\addbibresource{thesis.bib}
%% Abstract.
\thesislong{abstract}{%
TBA: abstract
}
%% Thanks.
\thesislong{thanks}{%
TBA: thanks, TBA: mention RH and MU projects
}
%% Index.
\usepackage{makeidx}      %% The `makeidx` package contains
\makeindex                %% helper commands for index typesetting.
%% Compact lists.
\usepackage{paralist}
\usepackage{enumitem}
\setitemize{noitemsep,topsep=3pt,parsep=3pt,partopsep=3pt}
%% Mathematics
\usepackage{amsmath}
\usepackage{amsthm}
\usepackage{amsfonts}
\usepackage{amssymb}
%% Graphics
\usepackage{tikz}
%% Colors
\usepackage{xcolor}
\definecolor{theme-red}{rgb}{0.62,0.01,0.05}
\definecolor{dark-red}{rgb}{0.6,0.15,0.15}
\definecolor{dark-green}{rgb}{0.15,0.4,0.15}
\definecolor{medium-blue}{rgb}{0,0,0.5}
\definecolor{light-gray}{rgb}{0.93,0.93,0.93}
\definecolor{gray}{rgb}{0.5,0.5,0.5}
%% Source code highlighting
\usepackage{listings}
\lstset{
  backgroundcolor=\color{light-gray},
  frame=lines,
  rulecolor=\color{black},
  numbers=left,
  numberstyle=\tiny\color{gray},
  basicstyle=\ttfamily,
  showspaces=false,
  showstringspaces=false,
  escapeinside={<*}{*>},
  belowskip=0.2em,
  %identifierstyle=\color{black},
  %keywordstyle=\color{blue},
  %stringstyle=\color{teal},
  commentstyle=\itshape}
%% Subcaption package for subfigure environment.
\usepackage{subcaption}
%% URLs.
\usepackage{url}
\usepackage{hyperref}
%% References.
\usepackage{cleveref}  % Must be loaded after hyperref.
%% Include shared macros used across chapters.
%--------------------------------------------------------------------
% Page-wide centered image with caption.
%--------------------------------------------------------------------
% #1: proportion of textwidth to use [optional, default: 1]
% #2: image ID
% #3: caption
\newcommand{\imgW}[3][1] {
\begin{figure}[h]
  \begin{center}
    \includegraphics[width=#1\textwidth]{img/#2}
  \end{center}
  \caption{#3}
  \label{fig:#2}
\end{figure}
}


% Shrink space between figure and caption
%\setlength{\abovecaptionskip}{3pt plus 3pt minus 2pt}
\begin{document}

\chapter{Introduction}
\label{chap:introduction}

%\itquote{
%Enjoyment appears at the boundary between boredom and anxiety,
%when the challenges are just balanced with the person's capacity to act.
%}{Mihaly Csikszentmihalyi}

\itquote{
To prepare humanity for the next 100 years, we need more of our children to learn computer programming skills, regardless of their future profession. Along with reading and writing, the ability to program is going to define what an educated person is.}
{S. Khan}

Programming is becoming an essential skill to learn.
Not only is it useful in increasing number of professions,
but also it develops abstract thinking and problem solving ability. % cite
% + divergent thinking
Our mission is to help children to learn basic programming efficiently,
while supporting their motivation for further learning.
For this purpose, we build a personalized system for learning introductory
programming, RoboMission%
\footnote{English localization available at \url{en.robomise.cz}.}
(\cref{fig:robomission-task1}),
that aims to adapt to the skills of the students
to create an optimal learning experience.

%How to build such a system? (make this paragraph an outline)
Many tutorials for introductory programming already exists,
including popular Hour of Code used by millions of students \cite{hour-of-code}.
These tutorials combine several strategies to support learning and motivation,
that are based on human needs, strengths, and weaknesses
\cite{lowering-barriers}.
However, they are not personalized and offer the same sequence of tasks
to everybody.
RoboMission improves upon the existing systems by adapting to the
students, using techniques of artificial intelligence,
which were already successfully used in other domains
\cite{mathsgarden, alg.evaluation-geography, matmat.response-times}
%, sqltutor}.
% maths += Cognitive Tutor (?mastery-learning-scale), ASSiSTments (Koedinger
% 2010), ALEKS
% TODO: find another domain (avoid another math)
%Artificial intelligence proved to be a powerful tool for tackling difficult problems, from playing chess to driving autonomous cars. % TODO: references
%This thesis explores how we can use artificial intelligence to optimize
%learning of introductory programming.
%This thesis explores how can be artificial intelligence used to optimize
%learning of introductory programming.


%In addition to the time-tested strategies and artificial intelligence,
%it also employs iterative design and improvements based on analyses.

\imgW[0.9]{robomission-task1}%
  {RoboMission is a web application for learning introductory programming.
   The left panel displays a space themed grid world. In the workspace on
   the right, students build programs using blocks describing actions
   such as \texttt{fly left} and \texttt{fly right}.
   The task is to create a program that would lead the spaceshipt to the last row.}
%  {RoboMission is a web application for learning introductory programming %
%   using space themed grid world and block-based programming.}

The adaptation is performed on two levels. % -- single student and the whole system.
First, the system estimates skills of the students as they solve tasks,
so that it can recommend them optimally difficult tasks to practice next.
By giving students tasks of difficulty matching their skill,
%neither too easy, nor too difficult,
we help them to achieve complete immersion into the problem solving
activity (\cref{fig:flow}).
This \emph{state of flow} \cite{flow}
is essential for the optimal learning experience
\cite{adaptive-practice}.
Second, using insight from analysis of collected data,
we iteratively improve behavior of the system,
e.g. moving tasks to more appropriate problem sets,
removing problematic tasks,
or adding new tasks practicing concepts that students often struggle with.


\begin{figure}[htb]
  \centering
  \begin{tikzpicture}[font=\sffamily,scale=5]
  \large
  %\draw [lightgray, fill=gray] (0,0) -- (0.1,0) -- (1,0.8) -- (0.8,1) -- (0,0.1) -- (0,0);
  \draw (0.1,0) -- (1,0.8);
  \draw (0,0.1) -- (0.8,1);
  \draw [thick, <->] (0,1) node [left] {\emph{difficulty}} -- (0,0) -- (1,0) node [below right] {\emph{skill}};
  \node at (0.27,0.82) {frustration};
  \node at (0.6,0.6) {flow};
  \node at (0.7,0.2) {boredom};
  \end{tikzpicture}
  \caption{
    Relationship between difficulty and skill.
    Too easy tasks bore students, too difficult frustrate them.
    For efficient learning, the difficulty of the task
    and the skill of the student should be in balance.
  }
  \label{fig:flow}
\end{figure}

%TODO: related areas
%- introductory programming learning
%- adaptive learning / intelligent tutoring systems
%- recommendation systems (with performance instead of ratings)
%- HCI, (software learnability)
%- games design \cite{book-of-lenses}

% EXT: Questions
%How to design a programming game that supports both learning and motivation?
%How to model the domain and students of introductory programming?
%% OR: How to estimate skills of the students and how to model learning of IP?
%Which algorithms to use for task recommendation and mastery learning?
%How to evaluate and iteratively improve components of the adaptive learning system?
%%How to evaluate if the adaptivity of the system helps to improve learning and engagement?
%% TODO: Make sure these questions are answered in the text (or remove them).
%To answer these questions, we not only look at the existing systems
%and research, but we also develop our own adaptive learning system for
%introductory programming, RoboMission\footnote{Available at \url{en.robomise.cz}}
%(\cref{fig:robomission-task1}).
%Development of this system helps us to better understand related challenges
%and allow us to collect data that we need to support or reject our hypotheses.
%% TODO: And are there such hypotheses in the thesis?
%% NOTE: RoboMission = app for efficient and engaging learning of introductory
%% programming for children (= mission)


In this thesis, we first look at how the introductory programming is taught
in existing online learning systems
(\cref{chap:learning-programming}),
and describe relevant techniques of adaptive learning
(\cref{chap:adaptive-learning}).
%We focus on models of domain and student for introductory programming,
%algorithms for task recommendation and mastery learning,
%and evaluation of the system and its components.
Then we present a new game designed for learning introductory programming
(\cref{chap:design-of-game}),
our approach to adaptivity (\cref{chap:design-of-adaptivity}),
and implementation of the system (\cref{chap:implementation-of-robomission}).
We conclude the thesis with analyses of collected data
(\cref{chap:analysis}).

\chapter{Learning Programming}
\label{chap:learning-programming}

% TODO: move the chapter descripton into intro/thesis-structure
This chapter describes the current state of the art of teaching programming, both from the view of successful learning systems and from the view of a research on learning programming.

\section{Existing Systems for Learning Programming}
\label{sec:existing-systems}

There are many systems for learning programming.
These systems can be categorized by several criteria:

\begin{table}[htb]
\centering
\begin{tabular}{l l}
\toprule
Criterion & Examples \\
\midrule
Tangibility & computer applications, physical toys \\
%Target group & primary school, high school students \\
  % Target group interacts with prerequisities
Prerequisities & reading, writing, mathematics \\
Content & loops, conditions, variables \\
Form & tasks, videos, texts \\ % worked-out examples,
Tasks & robot on grid, turtle graphics \\
% Playing-learning ratio & mostly playing, mostly learning \\
Programming language & block-based, textual \\
Adaptivity & task recommendation, mastery learning \\
%Price & free, paid
\bottomrule
\end{tabular}
\label{tbl:existing-systems-categorization}
\end{table}


The rest of this section describes the most notable systems.
Section \ref{sec:strategies-for-easier-learning} then extrapolates useful
strategies for teaching introductory programming.


\subsection{LightBot}
\label{sec:lightbot}
LightBot%
\footnote{Available at \url{http://lightbot.com/}.}
is a mobile and web application with a fixed sequence of tasks solved by
a block-based programming language
(figure \ref{fig:lightbot-robotanist}).
%(figure \ref{fig:lightbot-instruction}).
Students create simple programs to control a robot living in a grid world.
The robot can not only walk and turn left and right, but also jump and turn on lights.
Having five different basic commands is useful for diversity of elementary tasks.
The system covers sequences of commands, procedures, and simple loops via tail-recursion.

%\subsection{Robozzle}
%\label{sec:robozzle}
%With a robot on a grid and block-based programming, Robozzle%
%  \footnote{Available at \url{http://www.robozzle.com/}.}
%  is similar to LightBot;
%however, it adds a new feature: colored fields.
%Colors can be used in conditional commands,
%  which allows for hundreds of diverse tasks,
%  including difficult tasks with sophisticated recursion.
%Robozzle also allows users to create their own tasks,
%  which can be tried and rated by other learners.
%
%\imgW[0.8]{robozzle}{Robozzle.}


\subsection{Problem Solving Tutor}
\label{sec:problem-solving-tutor}
Problem Solving Tutor%
\footnote{Available at \url{tutor.fi.muni.cz}.}
includes a few problem sets for practicing programming,
such as Interactive Python, Karel the Robot, or Robotanist
% TODO: more precise description of the tasks instead of e.g. "Inter. Python",
% see the reffed paper for inspiration.
\cite{proso}.
Robotanist (figure \ref{fig:lightbot-robotanist}) is similar to LightBot
with an addition of colored fields.
Colors can be used in conditional commands, which allows for diverse tasks,
including difficult tasks with sophisticated recursion.  After
each solved task, Problem Solving Tutor shows a recommendation for two next
tasks, with predicted solving times.
Showing predicted time to the user serves as a motivational element -- it poses
a suitable challenge to overcome oneself
\cite{pelanek-student-modeling-times}.

%\imgW[0.6]{lightbot}{LightBot.}
%\imgW{lightbot-instruction}{LightBot provides clear and simple interactive instructions.}
%\imgW[0.6]{robotanist}{Robotanist in Problem Solving Tutor, Detour task.}

\begin{figure}[htb]
\begin{center}
  \includegraphics[width=0.485\textwidth]{img/lightbot-instruction}
  \includegraphics[width=0.47\textwidth]{img/robotanist}
\end{center}
\caption{%
  Left: Ligtbot.
  Right: Robotanist (Problem Solving Tutor).}
\label{fig:lightbot-robotanist}
\end{figure}




\subsection{Blockly Games}
\label{sec:blockly-games}
Blockly is a popular block-based programming interface.
In contrast to the blocks in LightBot,
Blockly blocks can be assembled in nested control structures
(figure \ref{fig:blockly-hoc}, left).
Blockly Games%
\footnote{Available at \url{https://blockly-games.appspot.com/}.}
% TODO: Replace by figure using nested control structures
% and move the fre to the end of previous sentence.
%(figure \ref{fig:blockly-instruction})
%demonstrate Blockly usage. The webpage ...
consist of several problem sets with Blockly, ordered by increasing difficulty.
For example, in the first level, students learn how to compose blocks together
as a puzzle, and in the second level they learn loops and conditions by solving
tasks in a maze. Final level serves as a transition from block-based
programming to textual programming in JavaScript.
Each level consists of 5-10 tasks, again ordered by increasing difficulty. %, with no personalization.
The fixed order enables to build on the program from the previous task,
thus gradually leading to more complex programs,
resulting for example in sophisticated images in turtle graphics
Blockly Games includes non-ignorable interactive instructions,
which require to take a described action before they disappear
\cite{blockly-10-things}.

%\imgW[0.7]{blockly-instruction}%
%  {Blockly Games includes non-ignorable interactive instructions, %
%  which require to take a described action before they disappear.}

\subsection{Hour of Code}
\label{sec:hoc}
Hour of Code%
\footnote{Available at \url{https://hourofcode.com}.}
%(figure \ref{fig:hour-of-code-sw})
provides many one-hour tutorials, each containing about 15 tasks in fixed order,
and using Blockly-based language
(figure \ref{fig:blockly-hoc}, right).
These tutorials focus on motivation, using themes from popular movies and
games, and providing videos with famous people explaining programming concepts.
The tutorials use high-level theme-specific blocks, such as ``set droid to a
[random] speed``, and they are restricted to only one or two programming
concepts, e.g. sequences and events.
Sometimes, the built program is not a direct solution for a robot,
but rather a game, in which the code specifies actions triggered on events.

%\imgW[0.7]{hour-of-code-sw}{Hour of Code, Star Wars tutorial.}

\begin{figure}[htb]
\begin{center}
  \includegraphics[width=0.45\textwidth]{img/blockly-nested}
  \includegraphics[width=0.52\textwidth]{img/hour-of-code-sw}
\end{center}
\caption{%
  Left: Blockly Games.
  Right: Hour of Code (Star Wars theme).}
\label{fig:blockly-hoc}
\end{figure}

\subsection{Human Resource Machine}
\label{sec:human-resource-machine}
Human Resource Machine%
\footnote{\url{http://tomorrowcorporation.com/human-resource-machine-hour-of-code-edition}}
%\footnote{Free Hour of Code edition available at \url{http://tomorrowcorporation.com/human-resource-machine-hour-of-code-edition}.}
is an example of an offline computer game for learning programming.
%Although it is presented as a game,
%the player spends nearly all the time solving tasks similar to those in the previously mentioned learning systems.
%The sequence of tasks is non-personalized and nearly linear,
%with only a few short side branches.
It also uses block-based programming, but in a different domain:
it offers low-level programming commands, such as
input, output, move-from, move-to, add, or jump
%(figure \ref{fig:human-resource-machine}).
(figure \ref{fig:hrm-ka}, left).
% input, output, move-to, move-from, add, sub, jump, jump-nonzero, jump-negative.
The game comes with a debugger, giving a possibility to step through the program.
All elements are explained when they first appear, and programming blocks can
be explained again simply by dragging the block onto a field with a question
mark.

%\imgWithFootnote[0.7]{human-resource-machine}{Human Resource Machine}%
%{Source: \url{https://tomorrowcorporation.com/humanresourcemachine}}
%%{Source: \url{https://tomorrowcorporation.com/humanresourcemachine} (Tommorow Corporation).}

\subsection{Khan Academy}
\label{sec:khan-academy}
Khan Academy has a computer programming curriculum%
\footnote{Available at \url{https://www.khanacademy.org/computing/computer-programming}.}
that uses textual programming in JavaScript with functions for drawing shapes
in absolute coordinates %(figure \ref{fig:khan-academy}).
(figure \ref{fig:hrm-ka} right).
In addition to programming tasks, it contains text and video explanations, and
projects. Some videos are in the form of interactive \emph{talk-throughs},
in which the student can fiddle with the explained code at any moment to understand
how it works.

%\imgW[0.8]{khan-academy}%
%  {Parting Clouds programming task on Khan Academy.}

\begin{figure}[htb]
\centering
\includegraphics[width=0.46\textwidth]{img/human-resource-machine}
\includegraphics[width=0.515\textwidth]{img/khan-academy}
\caption{%
  Left: Human Resource Machine.\protect\footnotemark~%
  Right: Khan Academy.}
\label{fig:hrm-ka}
\end{figure}
\footnotetext{Source: \url{https://tomorrowcorporation.com/humanresourcemachine}.}

%\subsection{We Can Code}
%\label{sec:umime}
%
%TODO: present UmimeProgramovat -> turtle graphics, mastery learning
%\footnote{Available at \url{https://www.umimeprogramovat.cz/} (Czech only).}


%\subsection{Ozobot}
%\label{sec:ozobot}
%Ozobot%
%\footnote{Product webpage available at \url{http://ozobot.com/}.}
%is a small physical robot
%programmable by either a Blockly-based interface,
%  or even without a computer
%  by drawing lines on a paper and reading them with a color sensor on the robot.
%As it does not require ability to read and write
%  Ozobot is suitable for even very young children.

%\imgWithFootnote[0.8]{ozobot-robot}{Ozobot}%
%{Source: \url{https://www.flickr.com/photos/robpegoraro/19464353039}, Rob Pegoraro.}

%\imgW{ozobot}{%
%  Ozobot simulator (available at \url{http://games.ozoblockly.com}) %
%  is a Blockly-like interface for creating programs for Ozobot %
%  (but the simulater requires the ability to read). %
%  It also includes a few prepared tasks.}


%\subsection{Project Euler}
%\label{sec:project-euler}
%The core of Project Euler%
%\footnote{Available at \url{projecteuler.net}.}
%is a list of several hundreds of programming problems with one correct answer.
%The problems can be solved in any language
%and then checked whether the computed answer is correct.
%The project is not meant to teach elementary programming,
%but rather to hone one’s programming skills;
%more complicated tasks even require some knowledge of algorithms and data structures.
%
%\imgW[0.7]{project-euler-progress}{%
%Project Euler provides several means of motivation: %
%levels (based purely on number of solved tasks), badges %
%(e.g. for solving 10 consecutive problems, 50 prime numbered problems, etc.), %
%comparison with user’s friends, %
%statistics page with several leaderboard (e.g. by country), %
%and a special score based on the performance on the most recently published problems.}


%\subsection{HackerRank}
%\label{sec:hacker-rank}
%
%Online programming contests,
%in which people attempt to solve as many problems as possible
%in limited time frame of a few hours,
%became popular in the last years.
%In addition to organizing such contests regularly,
%HackerRank%
%\footnote{Available at \url{https://www.hackerrank.com/}.}
%also provides many training programming tasks for various topics,
%from introductory programming to machine learning.
%% Solutions are evaluated on a server against a prepared set of test cases with time limits.
%In addition to the classic motivation in the form of points, badges and leaderboards, students are motivated to practice to perform well in the competitions,
%where they can win some prizes and sometimes even job offers.
%HackerRank also helps student to decide on a problem to solve by showing its difficulty according to the author of the task (on a 3-star scale),
%as well as success rate among the past submissions.
%Furthermore, after solving a task,
%a specific recommendation the next task to solve is shown.


%\imgW[0.6]{hackerrank}{HackerRank. %
%Each task specifies input and output format, including some examples. %
%Solution can be written either locally or in the provided online editor.}


\section{Strategies to Support Learning}
%\section{Strategies for Easier Learning}
\label{sec:strategies-for-easier-learning}

Learning programming is difficult,
  because it requires
  to adopt algorithmic thinking,
  understand program execution,
  and remember formal syntax of a programming language,
  all these three skills at once. % at the same time.
To make learning easier,
  the systems presented in the previous section use diverse strategies,
  such as avoiding syntax errors,
  displaying visual output
  and providing hints.
Various other strategies were tried in the past as well;
article \emph{Lowering barriers for Novice Programmers}
  \cite{lowering-barriers}
  provides a detailed overview.


\subsection{Avoiding Syntax Errors}
\label{sec:avoiding-syntax-errors}

A common strategy for avoiding syntax errors is to replace textual programming
with drag-and-drop block-based programming.
There are two basic types of block-based interfaces:
  either with a square grid defining the program shape,
  often one row per function
  (figure \ref{fig:lightbot-robotanist}),
  or the blocks can be nested and assembled arbitrary,
  with no limit on maximum program length,
  often one vertical stack per function
  (figure \ref{fig:blockly-hoc}).
Fixed grid is simpler to understand and manipulate,
  but it does not allow for nesting,
  which is a fundamental feature of computer programs.
This restriction is usually overcome by
  combining condition and a command into a single block,
  by using recursion instead of loops,
  and by replacing nested sequences of commands by a new function.

%TODO: the block-based editors are a special case of "structred editors",
%idea: editor only allows edits that transform the code from one syntactically correct
%program to another (basically working on AST level, edit = transforms to another AST)
%(compared to: classical text editors, where the edit can be arbitraty, making the code
%not syntactically correct).

%\subsection{Potential Drawback of Block-based Interfaces}
%\label{sec:potential-drawback-of-block-based-interfaces}
A drawback of using block-based programming
  is that the students need to learn a proper textual programming language in
  the future to be able to implement more complex programs.
Several controlled experiments were performed to test a hypothesis
  that it is still beneficial to start an introductory programming course
  with a block-based programming,
  even when the students will be writing textual code later in the course
  \cite{comparing-blocks-text-price2015, comparing-blocks-text-weintrop2017}.
Results suggest that block-base interfaces lead to increased learning and
motivation; however, the evidence is not fully convincing. For example, in
some of these studies the programming interfaces differed in more aspects than
just in using blocks instead of text. Furthermore, these studies do not
answer when it is the right moment to switch from block-based to textual
programming.

%\subsection{Transition Strategy}
%\label{sec:transition-strategy}
%Weintrop and Wilensky suggest that the block representation of the code
To make the transition easier, block representation of the code
  should match the underlying programming language
  to which the student is expected to move in the next phase
  \cite{challenges-of-blocks-based-environments}.
However, resemblance to a programming language sometimes
  conflicts with the readability of the blocks for novices.
Instead of making compromises,
  the system can progressively change the available set of blocks in each level,
  making them more similar to a textual programming language.
In the last level of Blockly Games, which employ this strategy,
  the text on the blocks matches the generated JavaScript completely
  \cite{blockly-10-things}.

%TODO:
%- note: "blending block-based and text-based programming approaches (e.g., Pencil Code
%(Bau 2015), Tiled Grace (Homer and Noble 2014), and Greenfoot’s Frame-based editor (Kölling et al.
%2015)"


\subsection{Visual Output}
\label{sec:visual-output}

Learning systems can help students to track program execution
  by providing a clear visual representation of the current state
  and effects performed by the program.
This can be achieved naturally for turtle graphics,
  where the whole state is just a position and orientation of the turtle,
  and effects are drawn lines.
Similarly, in simple games such as those described in
  \cref{sec:lightbot,sec:problem-solving-tutor,sec:blockly-games,sec:hoc}.
  the grid world visualization contains complete information about the current
  state (\cref{fig:lightbot-robotanist,fig:blockly-hoc}).
For the simplicity of their visual output,
  drawings and grid world games have become prevalent task types
  in the current systems for learning programming.

\subsection{Instructions and Hints}
\label{sec:instructions-and-hints}

Most educational systems include instructions
  to explain new concepts such as loops and conditions.
%However, Neil Fraser explains that students ignore instructions,
However, students ignore instructions,
  no matter how prominent they are \cite{blockly-10-things}.
A solution to this problem are actionable non-ignorable instructions,
  which cannot be closed manually by the student, and disappear only once the
  student perform the action described in the instruction
  (\cref{fig:blockly-hoc} left).

In additon to the instructions, some systems offer hints, which appear either
  upon a student request, or automatically after a certain time of unsuccessful
  solving. Although it is possible to generate a hint in any state,
  using data of students which have successfully solved the task before
  \cite{generating-hints}, the existing systems use a few manually prepared
  hints, intstead of relying on the automatic approaches.

\section{Strategies to Support Motivation}
%\section{Strategies for Motivation}
\label{sec:motivation}
% NOTE: Prev section: learning, this section: engagement.

In addition to strategies for easier learning presented in section
\ref{sec:strategies-for-easier-learning}, it is equally important to create an
engaging environment supporting students’ motivation.
All strategies for supporting motivation are based on fulfilling some human
nees \cite{nvc}. % TODO: cite specific page
\Cref{tbl:motivation-strategies} links needs to common strategies.
% TODO? related: flow, happiness, switch-elephant (prev. section: path)

\begin{table}[htb]
\centering
\begin{tabular}{ll}
\toprule
Needs & Strategies \\
\midrule
Purpose & Emphasizing usefulness of the programming skill. \\ %, confidence
Progress, learning & Skills visualization, points, levels. \\
Effectiveness & Recommending tasks of optimal difficulty. \\ % concentration/flow
Autonomy & Allowing to choose a topic or a task. \\
Recognition & Badges, leaderboards. \\  %Appreciation  % leaderboard vs. scoreboard?
Sharing & Possibility to share programs or achievements. \\
Cooperation & Pair programming. \\
Beauty, harmony & Appealing game world. \\ %, which is nice to look at and fun to play with \\
Fun & Entertaining tasks. \\
Creativity & Open-ended tasks, projects. \\  %, self-expression
% Community?
\bottomrule
\end{tabular}
\caption{Needs and strategies that help to fulfill these needs.}
\label{tbl:motivation-strategies}
\end{table}



\subsection{Appealing Game World}
\label{sec:motivation.game-world}
Ideally, game world should appeal to students --
even without tasks to solve,
  it should be an interesting toy to play with \cite{book-of-lenses}.
That is why all Hour of Code tutorials are based on movies and games
  popular among children, such as Angry Brids, Frozen, or Star Wars
  (\cref{fig:blockly-hoc} right).
In such environments, it is possible to assign open-ended tasks,
  or even let the students create whatever they want,
  which works well especially for creative students seeking for self-expression.
For example, Khan Academy programming curriculum contains many several
open-ended drawing projects (\cref{fig:hrm-ka} right).
%\cref{sec:robomission.game-world}

\subsection{Entertaining Tasks}
\label{sec:motivation.tasks}
For many students, giving them specific small problems works better
  than large, loosely defined, open-ended tasks.
By solving small problems quickly,
  they get a feeling of progress and learning.
Another advantage of closed tasks
  is a more straightforward implementation of gamification features and adaptive behavior.

Small closed tasks result in short programs,
  but their behavior should be still interesting. % for the students to be satisfied.
To achieve complex behavior,
  a system can either provide students with macro-commands (e.g. to draw a circle)
  or with a skeleton of complex code, with a few gaps to fill in by students.
However, it is important for the students to feel ownership over the code,
  which is especially a concern with the code skeleton.
Solution implemented in Blockly Games
  is to make a series of tasks in which the students
  build on their program from the previous task
  \cite{blockly-10-things}.

% TODO: some examples in the form of figures + descriptions
% TODO: other important aspects: variability


\subsection{Optimal Challenge}  % OR: "Optimal Difficulty"
\label{sec:motivation.challenge}
For a great learning experience,
  difficulty of the task must match the skill of the student.
If the task it too easy,
  the student is not challenged and gets bored.
If the task is too difficult,
  the student becomes frustrated and desperate.
On the other hand, if the task has appropriate difficulty,
  the student is likely to be challenged and focused
  (\cref{fig:flow}).
The complete immersion into the task the student is solving at the moment is called
  a state of flow \cite{flow}
  or a \emph{zone of proximal development} \cite{zone-of-proximal-development}.
Achieving the state of flow maximizes the learning outcome \cite{adaptive-practice},
  and even increases the long-term level of happiness. % TODO: find a source of this claim
\Cref{chap:adaptive-learning} describes techniques for estimating student’s skill
  and show how to use that estimate for task recommendation and mastery learning.


\subsection{Gamification and Progress Visualization}

Sense of progress and learning can be boosted by visualizations of
progress towards mastery in the current topic, solved tasks, completed problems sets,
or acquired skills.
\Cref{fig:progress-visualization} shows examples of progress bars from various systems.

% TODO: cite relevant research, open learner model
% NOTE: also helps to decide what to prectice next, while preserving student's
% autonomy (soft recommendation)

\begin{figure}[htb]
\centering
\begin{subfigure}{.48 \textwidth}
  \centering
  \includegraphics[width=.9\textwidth]{img/ka-skills}
\end{subfigure}
\begin{subfigure}{.48\textwidth}
  \centering
  \includegraphics[width=.9\textwidth]{img/hour-of-code-progress}
  \bigskip
  \vspace{1mm}
  \includegraphics[width=.9\textwidth]{img/umime-progressbar}
\end{subfigure}
\caption{%
  Progress bars: Khan Academy topics, Hour of Code tutorial,
  and mastery progressbar in Umíme Programovat (``We Can Code'').}
\label{fig:progress-visualization}
\end{figure}


%\subsection{Gamification}

Although the programming tasks themselves can be considered as a game
(e.g. Human Resource Machine from \cref{sec:human-resource-machine} is
presented purely as a game),
most learning systems add further gamification elements to increase the sense
of progress.
Common gamification elements include points, levels, badges, and leaderboards.

% TODO: figures: Project Euler, KA
% TODO: ref relevant research

\chapter{Adaptive Learning}
\label{chap:adaptive-learning}

TBA: intro to adaptive learning


\section{Student Modeling}
\label{sec:student-modeling}

TBA


\section{Task Recommendation}
\label{sec:task-recommendation}

TBA


\section{Metrics and Evaluation}
\label{sec:metrics-and-evaluation}

TBA


\section{Iterative Improvement}
\label{sec:iterative-improvement}

TBA

\chapter{Design of Programming Game}
\label{chap:design-of-game}

\itquote[1mm]{
A game is a problem-solving activity, approached with a playful attitude.
}{J.\,Schell}

%% Terminology + importance.
Students learn programming by solving programming tasks.
% and students spent nearly all the time in the system by solving them.
Therefore, tasks are the key component of a system for learning introductory
programming, and the design of a programming game deserves a careful attention.
Similarly to the design of adaptive behavior, design of game is also an iterative
process of prototyping and testing new ideas.
Good programming games are a result of many gradual improvements
(\emph{rule of the loop}) \cite{book-of-lenses}.

%...since...
% TODO: terminology: game, task, PS
% NOTE: rule of the loop + strategy: prototyping several ideas, measure
% metrics to see which games work best.

%% Requirements.
The game should allow for tasks practicing all basic
programming concepts, such as sequence of commands, loops, and conditional
statements. % repeat, while, if, if-else, simple tests (comparing)
To enable outer-loop adaptive behavior, multiple diverse tasks practicing the same
concepts are needed, including a lot of tasks using only sequences of commands
without any advanced programming construct.
To support engagement, tasks must be fun and immediately appeal to be
solved.  % TODO: Fix English.
%(TODO: mention how we underestimated the importance of the entertaining tasks in the first prototype)
% (TODO?: other requirements mentioned in the prev. chapters (+REF?)

% TODO: Ideally, mimic and refer to the sections in the chapter on Learning Programming
% (how we incorporated the strategies for easier learning and for motivation)

We have designed a game which is a variation on a robot
on grid with a space theme, and which uses Blockly (\cref{sec:blockly-games})
for building programs
(\cref{fig:robomission-task2}).
These choices support all main strategies for easier learning of programming
(\cref{sec:strategies-for-easier-learning}),
including
avoiding syntax errors by using block-based programming, and
showing a visual output (the grid world).
% and providing short instructions and explanations. % TODO: elaborate?
We combine several strategies to support motivation (\cref{sec:motivation}),
such as appealing game world, entertaining tasks, progressing through levels,
and recommending tasks of optimal difficulty
(adaptivity discussed in \cref{chap:design-of-adaptivity}).
% TODO: target group: Currently, the system primarily targets at children
%  between 10-15 years? (or simply primary school)

\imgW[0.73]{robomission-task2}{%
  Example of a task with the space themed grid world.}

% NOTE: rule of the loop \cite{book-of-lenses}

\section{Game World}  % Space World
\label{sec:robomission.game-world}

The game world itself should be pleasure to look at and fun to play with,
even without a specific task to solve \cite{book-of-lenses}.
We have based the game world on a popular choice of a robot on grid,
using a theme of a spaceship flying through space and collecting diamonds
(\cref{fig:spaceworld}).
% TODO: better word than "popular"?
%% Game elements.
Each field in the \emph{space world} has a background color, which
the spaceship can read and use for decisions (e.g. turning left on red fields).
In addition to the spaceship controlled by the student,
there are the following game objects:
% there are a few other game objects: %, that can reside on one of the fields.
%These are
diamonds, that need to be collected,
large asteroids, that destroy the spaceship if it hits them,
small meteoroids, that can be destroyed,
and wormholes that serves as teleports.

The spaceship start on the bottom row and it always moves one row forward
after any action. % towards its final destination (the last row).
% TODO: elaborate / illustrative figure showing a step and a path
In addition to flying and turning, the spaceship can also shoot small meteoroids
(\cref{fig:spaceworld-meteoroids}).
Four basic actions (fly, left, right, shoot) already allows for a
diversity of simple tasks which only practice sequence of commands.
% TODO: limited energy, in order to force thinking about whether to fly or shoot
The spaceship has two sensors, one for the color under the spaceship, and
second for its horizontal position (column index). Having two different sensors allows
for diverse tasks practicing conditions, including testing inequalities, and
potentially also compound conditions.

A novel feature of the game is the default forward movement,
%where turning results in a shift to a neighboring column
%and each action (including turning) are linked with a forward movement.
which results in significantly shorter programs.
For example, to fly around a stone, only two commands are needed % (left, right),
instead of 8 (or 4 if the available commands include
movement in any of the four directions without turning)
(\cref{fig:spaceworld-path}).
% 4: LFFR (e.g. in StarWars game), 8: LFRFFRFL
% TODO: or 4 as a semi-step (single direction OR default movement)
Furthermore, as the spaceship is always facing up, a common left-right
confusion \cite{blockly-10-things} is mitigated.
To avoid too long worlds, we introduced wormholes, that teleport the
spaceship back, to reuse rows multiple times
(\cref{fig:spaceworld-wormholes}).
% (more possible remedies available options available)


%\item movement -- always 1 row forward ("continuous flight ahead") --
%  advantages:
%  (1) mitigates ubiquitous left-right turning confusion;
%  (2) shorter programs (REF example);
%  disadvantages:
%  (1) this behavior is different than what most users initially expect
%  (2) underutilization of fields (only 1 field in each row and only once)
%      (leading to long worlds).
%\item chosen remedy for the 2nd disadvantage: worm holes
%  % (another possibilities: multiple spaceships, long worlds)

%\begin{itemize}
%\item original version - robot-in-maze, issues:
%  \begin{itemize}
%  \item not flexible and did not allow for a lot of diverse easy tasks,
%  \item which is crucial for adaptive learning system
%  \item unnecessarily long programs for simple ideas (non-elegant programs) (TBA: show comparision)
%  \item non-elegant tasks
%  \item ugly worlds
%  \item not very original and entertaining for kids
%  \end{itemize}
%\item requirements (on the topic/theme/world):
%  \begin{itemize}
%  \item entertaining for children
%  \item allowing to create plenty of diverse (and entertaining) tasks, inluding very simple ones
%  \item programs should be not too verbose
%  \item not limiting adaptability (e.g. story requring a fixed sequence of tasks would be a problem)
%  \item (not limiting another topics later)
%  \end{itemize}
%\item fully observable world
%\item no chance (randomness would complicate evaluation and interpretation
%  of data) (there is still some \emph{surprise} in the game for the players:
%  ``Will my program solve the task or not?'')
%%\item REF to formal grammar for the space world in the next chapter
%\end{itemize}


\begin{figure}[htb]
\centering
\begin{subfigure}[t]{0.3\textwidth}
\centering
\includegraphics[height=49mm]{img/spaceworld-meteoroids}
\caption{Shootable meteoroids}
\label{fig:spaceworld-meteoroids}
\end{subfigure}
\begin{subfigure}[t]{0.35\textwidth}
\centering
\includegraphics[height=49mm]{img/spaceworld-path}
\caption{Path around asteroids}
\label{fig:spaceworld-path}
\end{subfigure}%
\begin{subfigure}[t]{0.33\textwidth}
\centering
\includegraphics[height=49mm]{img/spaceworld-wormholes}
\caption{Diamonds, wormholes}
\label{fig:spaceworld-wormholes}
\end{subfigure}
\caption{Examples of Space Worlds.}
\label{fig:spaceworld}
\end{figure}


%\imgW{prototype-pits}{Lesson from the first prototype: world must contain
%elements that the program cannot check to enable diverse tasks. As we had a
%natural sensor to check walls, we needed to add pits that could not be checked
%by any sensor.}


\section{Tasks and Programs}
\label{sec:robomission.programs}

Each task asks the student to create a program in Blockly,
that would guide the spaceship safely to the last row,
collecting all diamonds on its way. % (\cref{fig:tasks-solutions}).
% Advantage: very clear and intuitive goal
%Setting of a task consists of a space world description, % as described in the previous section
%optional limits (e.g. maximum length of the program), %, number of shots),
%and available programming blocks (\emph{toolbox}).
Students can execute their current programs as many times
as they need. % wish, and edit them until they solve the task.

%% Blocks.
The set of available blocks is gradually expanding with
increasing level (\cref{sec:level-design}). This \emph{toolbox standardization}
is easier for both task creators (it is enough to specify
a toolbox only once for all tasks in the same problems set,
instead of naming all available blocks for each task),
as well as for  students
(available blocks are not changing chaotically with each task).
%- available blocks: depending on the level (sweet spot between chaotically changing toolbox with each task and overloading by all commands from the beginning
% NOTE: blocks: actions (...), sensing (...), repeat, while, if, if-else;
The easiest tasks use only commands for actions (fly, left, right, shoot),
while the more advanced tasks offer repeat loop, while loop, if and if-else statements,
and tests for colors and position.
No special test for the last row is needed, since all tasks follow a convention
of coloring the last row by blue color.

%% Programs
In order to simplify future learning of a textual programming language,
labels on blocks approximately match Python syntax. % translated to Czech in Czech localization
The most notable exception is the repeat block (\texttt{repeat 5}),
where the Python equivalent (\texttt{for i in range(5)}) is more difficult
to understand for beginners.
% (For details about the blocks and code, see \cref{sec:robocode}.)
% NOTE: in future, this allows for AB experiments
%comparing directly the influence of the text/block environment as done e.g.
%in the field study in \cite{comparing-blocks-text-weintrop2017}.

%% Limits.
To force students to use loops instead of a long sequence of actions,
a task can specify \emph{length limit}
on the maximum number of statements in the program.
Limit on the number of statements rather than blocks was chosen in
order to make the limit a smaller number and thus the counting easier.
(Another advantage is that this definition could be used for textual
programming as well.)
To force students to think when they need to shoot, a task can specify
\emph{energy limit} on the number of shots.

%%% Limits.
%There are two limits that a task can specify.
%The \emph{length} limit defines maximum number of statements
%in the program. Limit on number of statements instead of blocks was chosen in
%order to make the limit a smaller number and thus the counting easier.
%(Another advantage is that this definition could be used for textual
%programming as well.)
%% which is not case for the blocks limit
%This limit forces students to use loops instead of a long sequence of
%actions.
%The \emph{energy} limit controls for the number
%of performed shots. Without this limit, students could always use
%shooting without thinking when it is really needed.


\begin{figure}[htb]
\centering
\begin{subfigure}[t]{0.5\textwidth}
\centering
\includegraphics[height=60mm]{img/robomission-task-repeat}
%\caption{Path around asteroids}
%\label{fig:spaceworld-path}
\end{subfigure}%
\begin{subfigure}[t]{0.5\textwidth}
\centering
\includegraphics[height=60mm]{img/robomission-task-conditions}
%\caption{Diamonds, wormholes}
%\label{fig:spaceworld-wormholes}
\end{subfigure}
\caption{Examples of tasks and corresponding solutions.}
\label{fig:tasks-solutions}
\end{figure}


\section{Level Design}
\label{sec:level-design}

% TODO: explore and cite: ``level design''

RoboMission contains over 80 tasks divided into 9 levels.
% (\cref{fig:robomission-tasks-overview}).
The levels were initially created manually, based on the required blocks
and an estimated difficulty. Later, after some performance data
were collected, we further decomposed them into a two-level
hierarchy to enforce prerequisites between tasks, and we moved
a few tasks into levels that better match their observed
difficulty for students.
% NOTE: But the observed difficulty depends on the context, such as
% the level, the task is presented in.

\imgW[1.0]{robomission-tasks-overview}{%
Tasks overview shows tasks grouped by levels.
RoboMission contains 9 levels, each with about 9 tasks.
Colors: green tasks are solved, the orange task is recommended.
}

As a motivational element that helps to reinforce the sense of progress,
% OR: to satisfy the need of progress,
students receive credits for each solved task
(\cref{fig:robomission-levels-credits}).
After earning a sufficient number of credits, students progress to next level,
which results in increased difficulty of recommended tasks.

\imgW[1.0]{robomission-levels-credits}{Students earn credits for each solved task.}


\section{Instructions and Explanations}
\label{sec:game.explanations}

Clear instructions and hints at appropriate moments
are another strategy to support learning
% OR: to make the learning easier
(\cref{sec:instructions-and-hints}).
We combine simple adaptive instructions and reflexive post-event explanations.
When a student encounters a new programming concept or a game element
for the first time, the system displays a short instruction
(\cref{fig:robomission-mini-instruction}).
% NOTE: They are adaptive in the way that they only show the first time student encounter
% the concept.

\imgW[0.81]{robomission-mini-instruction}{Mini instruction for a new game concept.}

The system also provides short explanations after each unsuccessful execution,
describing why the task was not solved,
e.g. some diamonds were not collected,
or the spaceship has not reached the final row
(\cref{fig:robomission-mini-explanation}).
% NOTE: simple inner-loop reflexive agent

\imgW[0.81]{robomission-mini-explanation}{Mini explanation of an unsuccessful attempt.}
%\imgW{prototype-instructions}{Instructions in the first prototype. Nobody was reading them. Most people even did not notice there are any instructions.}

\section{Game Editor}  % OR: task editor?
\label{sec:robomission.task-editor}

The online Game Editor (\cref{fig:game-editor})
helps to create new tasks easily.
In the editor, the author of a task can immediately see visualization of the
space world,
test a solution in either Blockly or its text-based equivalent %(\emph{RoboCode})
(\cref{fig:game-editor-vim}),
and import or export tasks (using a custom Markdown-based format).
%(TODO: mention how inconvenient it was in the first prototype, especially the tokens...)
The editor is public, so even students can create their own new tasks.
%However, there is currently no support for sharing the tasks with other students.
% Which support the need of creativity; + potentially allows to get a lot of tasks into
% the system, which is could be interesting for adaptivity.

% TODO:
%It allows to switch between RoboCode and RoboBlocks anytime. (REF:transformations)

\imgW[0.86]{game-editor}{Game Editor includes a text editor for game world
(bottom right), which uses an intuitive text representation described in
\cref{sec:impl.spaceworld}.}
\imgW[0.86]{game-editor-vim}{In Vim mode, Game Editor allows to perform some complex
modifications, such as replacing all yellow fields on selected rows to red color,
with a single command.}

\chapter{Design of Adaptivity}
\label{chap:design-of-adaptivity}

In the last chapter, we have mentioned how we incorporated many of the
strategies to support learning and motivation into the game.
In this chapter, we discuss another strategy which helps to achieve the state of
flow, the one that is special to ALS  % TODO:reformulate
-- recommending tasks of optimal difficulty.
% ... "which is the secret sauce of adaptive learning systems"
% ... "which is powered by the AI"
% TODO: More links to learning/motivation \label{sec:motivation.challenge}
% TODO: link to the adaptivity chapter

\section{Expected Behavior}  % or "Specification", "Goals", "System Behavior", "Behavior specification"
\label{sec:robomission.behavior}

% NOTE: This section should provide a useful basis for:
% - guide how to design ALS components
% - setting rewards for a a RL/planning agent
% - defining "good recommendations" (~good performance) for supervised
%   learning and for use in monitoring/evaluation (domain metric)

% TODO: clarify the relationship between this section and AL.analysis (or
% possibly also this-chapter.analysis) sections.

Wide range of task difficulties, combined with the adaptive behavior,
should make the system useful for anybody who wants to learn
introductory programming.
However, this long-term goal requires many iterations and a lot of data.
Currently, the system primarily targets at children 10-15 years old.
% "between 10 and 15 years."

The main use case of the system is a 1-2 hours in-classroom tutorial
(``Hour of Code'' style), with a possible follow-up individual practice at home.
In line with the system mission (\cref{sec:mission}), goals of this tutorial
are to teach the student basic concepts of programming,
help them to spend most of the time in the state of flow,
and motivate them for the further learning of computer science.
% TODO: better link to fullfilling human needs (as stated in table in chapter 2)
% TODO: better terminoglogy, because "You cannot motivate people... (motivation
% goes fron inside)"
% NOTE: flow is a goal; but at the same time it supports the other 2 goals

After each solved task, the student is shown a dialog window with one
recommended task, and with a link to the page with overview of all tasks.
% Terminology: \emph{good recommendation}
Ideally, each recommendation should lead to the state of flow.
% TODO: How to measure? One way: looking ath the long-term objectives
As we cannot directly observe whether the student is in the state of flow,
we must rely on proxy data, which can be either subjective
(using perceived difficulty ranking),
or objective (using observed performance), %should be neither too low, nor too high),
% "qualitative" vs. "quantitative"
% TODO: REF:performance
as we have discussed in section
on attributable metrics (\cref{sec:live-evaluation}).
% long-term or attributable metrics \cref{sec:long-term-objectives,sec:live-evaluation}
% TODO: the performance compression should be described in this chapter
% in student modeling section, ideally with an analysis supporting the
% decsions or sketching some possible directions.

% TODO: Reformulate: (e.g. why is it useful, useful for what)
Unfortunately, both subjective and objective approach to measure flow
is very noisy. Therefore, it is useful to formulate weaker, but better observable
requirements on the system behavior:
% NOTE: It reminds me of hypothesis testing of goodness-of-fit: there are
% various tests (metrics we can measure), that can tell us, that the given
% sample is definitely not from the compared distribution; some of these
% tests are very simple and intuitive (e.g. for exp. distribution, ratio
% of mean and stdev^2 should be 1), but they also have quite small power
% (i.e. they are often not able to reject false hypotheses).
% TODO: shrink vertical spaces
\begin{itemize}
\item The student is able to solve any task recommended by the system, ideally
  in a reasonable time (20 minutes).
    Furthermore, the student is able to solve the first few tasks quickly
    (in 2 minutes) and progress to the second level quickly (in at most 10
    minutes).
\item The best performing students (supposedly with a prior programming skill)
should progress through a first few levels quickly, spending at most 5 minutes
on tasks practicing only sequences of commands, and get to the more
challenging tasks containing both types of loops and conditional statements
in at most 20 minutes.  % 10, 15?
\item All students should gradually start practicing all basic programming concepts
  (sequence, loops, conditional statements) during the first hour of the tutorial.
  % TODO: check gvid
\item Average difficulty of the recommended tasks should gradually increase.
  (Occasionally, an easy tasks might be recommended in order to improve
  exploration, but these easy tasks should
  not be too frequent, and they should be differentiated for the student,
  e.g. as \emph{speed challenges} in order to explain the sudden change in
  difficulty.)
\item At most one new programming concept and one new game concept should appear
  in a task. No new concept should be introduced, until the student
  solves at least one task with the last introduced concept with a good
  performance. % (they must sometimes come together, it is not a problem)
% TODO: perform analysis if it's the case in the current system.
%Each new concept (e.g. block) is explained in the task where it appears for the first time. (REF: img), so the student understands all elements in the game world and blocks available in the toolbox.
\item If asked after each task, students should report that the task was
  neither too easy, nor too difficult for them in at least half of the tasks
  during the tutorial.  % or another negative tag, such as boring, weird, ...
\item Neither task that was reported as too easy should take them
  more than 1 minute, and there should never be more then 3 too easy tasks
  in a row (with the exception of the first level).
  %(Student is not bored by neither too easy task requiring many commands (i.e.
  %taking more than 1 minute to build the streightforward solution), nor by the
  %sequence of either too easy or too similar tasks.
  % TODO: Maybe it would be better to state it in combined max too-easy time?
  % NOTE: It is ok to have sometimes an easy task (esp. at the beginning), but
  % they should be quick to solve and not too frequent
\item If asked at the beginning and at the end of the tutorial about their
  interest in learning computer science, the end report should be statistically
  higher.
% TODO: Not sure if to mention the self-report points, as they are not
% currently used in the system.
% TODO: (Another flow-related) If the student marks the current task as too
% easy, the next task is strictly more difficult. If the student marks the
% current task as too difficult, the next task is as most as difficult as the
% current one.
\end{itemize}
% TODO: compare these requiremnts againts collected data and gvid-report



%\subsection{Main Use Case}
%\label{sec:robomission.use-case}
%
%\begin{itemize}
%\item Student visits the home page of the project, reads the "promotional slides" and tries the game with manual controls. On the last slide, they click on the recommended task from the 1st level.
%(REF: img)
%\item Student creates the program using Blockly blocks and can run the program as many times as needed. (REF: img) Program execution is visualized. Student can change the speed of the execution.
%\item After each unsuccessful execution, a short message explaining why the task was not solved is shown (e.g. "The spaceship must reach the final row.") (REF: img)
%\item Student is able to solve the first few tasks quickly (within 2 minutes).
%\item After solving each task, student is shown a visualization of obtaining points (called credits) (REF: printscreen). After a few solved tasks, student progresses to next level.
%\item After each solved task, student is shown a dialog with one recommended task, and also a link to the page with overview of all tasks (REF: img).
%\item Student is able to solve any task recommended by the system (within 15 minutes).
%\item Each new concept (e.g. block) is explained in the task where it appears for the first time. (REF: img), so the student understands all elements in the game world and blocks available in the toolbox.
%\item Students with some prior programming skill should progress through first few levels quickly (within 10 minutes) and get to the more challenging tasks containing both types of loops, conditionals etc.
%\item Student is not bored by neither too easy task requiring many commands (i.e. taking more than 1 minute to build the streightforward solution), nor by the sequence of either too easy or too similar tasks.
%\item Student can sign up (or log in without registration through their social accounts) at any moment to save their progress. Even without singing up, the system can associate the student with its progress using session cookie (but also provides a button to clear the history).
%\item Student can provide a feedback or report a bug easily (and the feedback is send to admins by an email).
%\end{itemize}
%(TBA: add diagram with images for all these steps linked by arrows showing transitions)

% TODO: consider some of the following notes
% - Design of tasks for the system is described in section \ref{sec:robomission.tasks}.
% - Adaptive aspect of the behavior is described in section \ref{sec:robomission.adaptability}.
%\item intuitive and simple user interface crucial (aiming at children, they need to focus on learning programming, it would be bad to waste their mental power on understanding a complex interface)
%\item mini-instructions (ref to the Google research on ignoring instructions, show how it was solved in Blockly Games; ref figure)
%\item mini-explations (difference from instructions: after the fact) (ref figure) (they also serve as a convenient mean to game resetting)
%\item motivation: intrinsic (fun challenging game + optimal difficulty) and simple external motivation scheme: credits and levels


%\subsection{Four Modes of Usage}
%\label{sec:robomission.use-cases}
%
%In addition to the main use case described in the previous section,
%which assumes a new student without any context (e.g. a classroom),
%the system can be potentially used in other (or more specific) ways.
%
%\begin{itemize}
%\item "Hour of Code" mode
%  \begin{itemize}
%  \item single hour
%  \item mainly as a motivation to programming
%  \item using RoboBlocks
%  \item directly at elemenatry and high schools, or at home
%  \item plus: MjUNI workshop
%  \item shorter promotianal version for DODs (?) ("10 minutes of code")
%  \item (should be strictly time-limited; certificate at the end)
%  \end{itemize}
%\item "Foundations mode" individual learning of elementary programming (individual at home or in a classroom, from several days to several weeks, depending on the prior skill); natural continuation of the first "Hour of Code" (next levels, with RoboBlocks)
%\item "University mode" -- levels with RoboCode/Python, at home / secondary schools, KSI (0th problem set), IB111 (0th/1st motivational lesson - needs Python and to be better than turtle)
%\item "Competition mode" -- competitions such as Purkiada, Pevnost FI, KSI
%(advanced problem sets), new FIBot (physical version already in InterSoB 2017,
%then in Sob 2018); this also includes testing mode for RH interns
%\end{itemize}
%
%All these modes can be naturally implemented as distinct levels,
%going from the easy tasks using RoboBlocks for "Hour of Code",
%gradually transitioning to the RoboCode during learning the "Foundations",
%using full-fledged Python for "University mode"
%and offering both blocks and Python for the individual competitions.
%Levels from the past competitions can be made public for all students.


\section{Domain Model}

%In this phase of development, we have decided not to model overlapping concepts,
% NOTE: not enough data, not clear how they combine and how they relate to PS
% which are still needed for the users
Currently, we do not model overlapping concepts, since their use would require
more data to analyze their interaction.
Instead, we devide tasks into disjoint linearly order hierarchical problem
sets (\cref{fig:robomission.domain}). % , which are easier to handle.
% ... We started by manually dividing tasks into 9 problem sets (levels)
The hierarchy has two levels: the top-level problem sets are call \emph{missions},
and they contain about 8-10 tasks, which are further split into three smaller
problem sets called \emph{phases}.
Missions and phases are ordered, while tasks within a given phase are not.
The missions were chosen primarily by the available programming blocks and
included game elements, and secondarily by an estimated difficulty.
% (initialy human-estimated, later refined using collected data
% NOTE: Data collected befor division into phases (TODO: mention in the
% analysis chapter...)
The contribution of refining the missions into phases is threefold:
\begin{itemize}
\item Phases enforce important prerequisities within a mission (e.g.
introducing wormholes before using them in more difficult tasks).
\item Phases help to achieve a balanced composition \cite{progression-analysis},
  introducing new concepts in the first phase,
  recombining them with previously learned concepts in the second phase,
  and further reinforcing (practicing) of known concepts in the third phase.
% TODO: Although this division is only a guide, doesn't hold exatly...
% TODO: elaborate on the relevance of the paper
% TODO: check the paper (terms of the phases, their "oreder" and meaning)
%The resulting two-level hierarchical and ordered problem set structure is shown
%in \cref{fig:robomission.domain}.
\item Phases are approximately homogeneous, i.e. difficulty of all tasks in a
  given phase is similar. This allows allows for simpler tutor models.
  % and they practic similar concepts
\end{itemize}
% TODO: link to what presented in earlier chapters

\begin{figure}[htb]
\centering
TODO: domain diagram\\
(all 9 missions with labels; unlabeled nodes for phases and tasks)
% should show covered concepts: sequences of commands, repeat, while, if, if-else, simple tests (comparing)
\caption{Domain model used in RoboMission.}
\label{fig:robomission.domain}
\end{figure}


\section{Student Model}

% TODO: terminology and links from \cref{sec:student-modeling}
For each student we keep track of her skill for each problem set. % phase and mission.
The skill $s \in [0, 1]$ represents manifested ability to solve tasks in
a given problem set. The initial skill is 0, it is increased after each solved
task from the problem set, and once the skill reaches 1 the problem set is solved.
The model is decsriptive, because it does not model the probability of
a student solving a task (with some performance), but rather an amount
of verified skill.
An advantage is that this skill can be shown directly to the student
(e.g. visualized as a progressbar towards completion of the current mission).
% E.g. at the beginning, the probability is higher than 0 for sure.
% used directly in visualization

We use discrete representation of performance with 3 levels (poor, good,
excellent). Currently, only solving time is considered for the performance
compression: for each task, we set a time threshold for a solution to be
considered as good or excellent
(set as 1.5 and 0.75 multiples of median time).
% TODO: support the decision by an analysis (REF)
% TODO: Note that it's huge simplification and a subject of future research.

After each solved task session, the corresponding skill is increased by
an amount $p \in [0, 1]$ which depends on the performance.
For each performance level, we set the increment such that $1/p$ tasks solved
with such performance are enough for manifesting mastery in this phase.
The increment does not depend on the task, because we assume homogeneous
phases.  %, and it does not depend on our previous estimate
Either a single task solved with an excellent performance, or two
tasks solved with a good performance are enough to solve the phase,
translating into $p_{excellent} = 1$ and $p_{good} = 0.5$.
Furhtermore, solving all tasks in a phase (even if with a poor performance),
should be always enough to solve the phase.
%(this is a technical requirement, stemming from the fact that we do not want
%to present students with the same task they already solved again)
Therefore, the update rule is:
$s \leftarrow \min(1, s + \max(p, \frac{1}{n}))$,
where $n$ is the number of tasks in the phase.
% The $1/n$ term ensures that the student eventually masters the phase.
% NOTE: we never want to decrease skill of a student

To aggregate the skills of the phases into the skill of the mission,
we simply take average of the phase skills.
In future, we would like to use more sohpisticated aggregation,
which would allow to compensate unsolved tasks in the first phase
by more difficult tasks in the later phases.
% TODO: reformulate

\section{Tutor Model}

We currently focus on the outer-loop tutor modeling.
Concerning the inner loop, the system provide only a basic
rule-based agent for displaying instructions and explanations,
as described in \cref{sec:game.explanations}.

The outer-loop tutor is hierachical;
it first selects a problem set, then a task from this problem set.
In order to assess if a problem set is mastered,
the corresponding verified skill
computed by the student model described in the previous section
is compared to the threshold of 1. % ?
% TODO: diagram of PS, TS selection and mastery decison

The tutor selects for practice the first unmastered phase of the first
unmastered mission, using the total ordering provided by the domain model.
The system contains only a small number of problem sets,
so it is reasonable to assume that the student would like to solve all of
them,
% TODO: further increased by the tendency to "get everything green"
in which case it is preferable to solve them in the order from the
easiest to the most difficult.
Homogenity of phases allow to safely select a task uniformly at random from
all unsolved tasks in that phase, which is a strategy maximizing exploration.
% UI:
Recommendations provided by the system are \emph{soft};
i.e. that student can ignore them and select any task
from the overview of all tasks.

Progression through the missions and phases gradually increases the difficulty,
but the increase is not monotonous.
% TODO: Reformulate not to use so many "increase"
Perceived difficulty increases after progess to a new phase,
but it decreases during solving tasks from the same phase,
as the student's skill improves, while the objective difficulty stays
approximately same (\cref{fig:robomission.flow}).
On one hand, it means that the difficulty does not always match student's
skills perfectly, possibly slightly overshooting at the beginning of a phase,
and undershooting at the end.
On the other hand, it means that the perceived difficulty level is not
same all the time, but has a wavy character, which is more interesting
for students % \emph{dramaturgy of flow}
\cite{book-of-lenses}.

\begin{figure}[htb]
\centering
TODO: flow diagram specifically for progress through phases
(increase of challenge with each new phase, then constant during the phase)
% result: waves, but more "spiky"
% (but the objective difficulty does not decrease, it stays constant
% (so it's important to think what is on the y-axis!)
\caption{Dramaturgy of flow in RoboMission.}
\label{fig:robomission.flow}
\end{figure}



% TODO? UI? Probably not enough material for this



\section{Analysis Layer}
\label{sec:robomission.analysis-layer}

In order to iteratively improve adaptivity, as well as the programming game
and other aspects of the ALS,
an analysis layer (\cref{sec:metrics-and-evaluation})
that provides several different views on the behavior of the system
is crucial.
Our analysis layer includes the following components:

% \subsection{Monitoring Components}
%To fulfill the requirements from section \ref{sec:admin-requirements},
%the system includes the following components:

\begin{itemize}
\item \emph{Google Analytics}
  shows distribution of users with respect to the time and space.
  In addition to page visits,
  it can also process specific events sent from the frontend
  (such as clicking on the execution button),
  and divide them into groups, e.g. by the task being solved
  % or group for AB experiment.
  (\cref{fig:google-analytics}).
\item \emph{Monitoring Dashboard}
  shows values of a few long-term objectives % metrics related to the long-term objectives
  %(\ref{sec:long-term-objectives})
  in the last month, namely daily active students, daily solved tasks, and
  solving hours (total time spent on successful attempts)
  (\cref{fig:monitoring-dashboard-fragment}).
  % + "success ratio" = proportion of successful attempts,
  The system also computes a few metrics for each task
  (solved count, median time, and success ratio) in order to detect
  bugs in them.
  % These metrics are recomputed every night.
\item \emph{Monitoring Notebook}
  is a weekly-recomputed jupyter notebook (CITE) that performs several analyses
  and visualizations using the latest data (\cref{fig:monitoring-notebook}).
  It serves the similar purpose as the monitoring dashboard, but it is
  easier to extend. It is enough to add a cell into the notebook;
  %and test it on a collected data,
  no change to the backend or frontend is needed.
  %NOTE: + investigation notebook that with a command that generates exports
  %and serves them directly as pandas data frames
\item \emph{Error Reports and User Feedback}.
  If an unhandled top-level error occurs on the server,
  it is not only logged, but also sent to the administrators.
  Administrators also receive emails with messages provided by users via
  a feedback form that can be invoked on any page.
\item \emph{Data Exports and Logs}.
  All collected data are exported every week as a zip bundle containing
  CSV files prepared for %convenient offline analysis.
  offline analysis.
  In addition, all requests, performed actions, and unhandled errors % and submitted feedbacks
  are logged to text files on the server for manual inspection.
\end{itemize}


\imgW{google-analytics}{%
  Preview from Google Analytics (January--March 2018).} % (breakdown for "execution" event).}

% TODO: make it readable or remove
\imgW{monitoring-dashboard-fragment}{%
  Number of active students and number of solved tasks during March 2018
  visualized in the monitoring dashboard.}

% TODO: check if the table lines looks ok when printed
\begin{figure}[htb]
\centering
\begin{subfigure}{.48\textwidth}
\centering
\includegraphics[width=0.95\textwidth]{img/monitoring-notebook-levels}
\end{subfigure}
\begin{subfigure}{.51\textwidth}
\centering
\includegraphics[width=0.95\textwidth]{img/monitoring-notebook-repeat}
\end{subfigure}
\caption{%
  Mean success and median time for all missions (left),
  and tasks in the Repeat mission (right),
  visualized in the monitoring notebook
  from March 31, 2018. This simple analysis reveals e.g.
  that 7th mission (comparing) is probably more difficult than the 8th (if-else),
  % it's not clear, because the population of students is different; they
  % might already learn something important in the 7th level for the 8th
  % level
  and that Clean Your Path task is significantly more difficult than
  the other tasks in the mission.}
% TODO: observation about clean-your-path
\label{fig:monitoring-notebook}
\end{figure}

% TODO: update investigation notebook (fix error in last cell, current data;
% should show something interesting, or at least some descriptive analylis)
%\imgW[0.7]{investigation-notebook}{Template of jupyter notebook for investigation of live data.}



% NOTE on iterative development: first prototype: 2016, one year of
% development, thrown away; for testing our initial ideas and find what works
% and what not; 50 tasks with a robot in maze; problems with the robot in the
% maze: not fun (boring, not inovative, repetitiveness), did not allow for a
% plenty of diverse easy tasks (which is necessary for adaptive systems),
% required a lot of blocks even for simple programs (compared to the SpaceGame)
% + problems with the codebase (maintainability, extensibility - why?); and the
% good things? (SPA, explored/verified useful technologies, such as Blockly and
% Django)

%\imgW{prototype-task-environment}{First prototype of the system, with a classic robot-in-maze game.}


%\subsection{Admin Requirements}
%\label{sec:admin-requirements}
%Similarly to regular users, administrators also have requirements on the systems:
%\begin{itemize}
%\item Admin can immediately see how much is the system used and how the system behaves with respect to the short-term ("live-evaluation") metrics (...)
%\item Admin receives feedback from provided by users, and error reports on unhandled exceptions.
%\item Admin can see metrics on individual tasks (to quickly detect issues with a task).
%\end{itemize}


% TODO: consider if to include non-functional requirements or not
%\section{Non-functional Requirements}
%\label{sec:robomission.nonfunctional-requirements}
%
%\begin{itemize}
%\item easy to understand code, pleasure to read and write (extend)
%\item easy to refactor and add new things (new tasks, levels, recommendation strategies etc.)
%\item robust, efficient, interpretable behavior
%\end{itemize}

\chapter{Implementation of RoboMission}
\label{chap:implementation-of-robomission}

This chapter describes general architecture of the RoboMission application,
together with a few specific aspects, such as representation of task sources,
and transformation between programming blocks and code.

\section{System Architecture}

The application uses a client-server architecture,
with a fat frontend client communicating with the server backend via REST API.
In addition to the frontend app and backend service,
there are two other parts of the system:
scheduled jobs, which run periodically every night (e.g. metrics computation),
and tools for offline analysis of collected data
(e.g. jupyter notebooks). % templates).
% TODO: and "helper functions"
% TODO: ref/cite human-in-the-loop principle
% TODO: consider: list the parts in itemize
% TODO: other potential parts (future):
% - tasks/content management tools (CLI+browser)
% - simulations (CLI+browser)
% TODO: consider to include FE and BE sections below as subsections of this section
%TODO: diagram of overall architecture (client-server, communication, monitoring, scheduled jobs, offline analysis)

% \section{Frontend}
Frontend is a single page application with \emph{redux architecture} (CITE),
which means that a single immutable state stores all application data.
The state cannot be mutated directly; a new state can be created only
%from the old one
by dispatching actions.
Each part of the state then defines its \emph{reducer},
which is a pure function that takes previous state and action, and returns a new state.
% ... and the only way the state can be changed is by dispatching actions.
% ... dispatcher then takes dispatched actions one by one and reduce them (notifying
%     connected containers about the new state)
% ... reducers: state, action -> new state (pure functions)
% "unidirectional UI" (as opposed to MVC and similar architectures)
The view is then assembled using declarative reusable components, that can
be either \emph{presentational} (e.g. describing how the game world should be
rendered), or \emph{behavioral}, selecting data from the state to use for
the rendering, and defining actions to dispatch and their triggers.
The application also defines a few asynchronous workflows called \emph{sagas},
e.g. for program interpretation, for task solving proces, or for sending events to
Google Analytics.  % better examples?
%% TODO: at least one sentence about React components?
%\subsection{React Components}
%
%\begin{itemize}
%\item mostly declarative - simple mental model: rebuilding from scratch every time anything change -> less error prone
%\item reusability -> use of single component on many places in different contexts,

%% TODO: Improve the diagram to make a point; otherwise remove.
%\imgW{frontend-dependencies}{%
%  Frontend modules/packages and dependencies between them.}
% TODO: Give examples of (in description) of what are submodules inside core, utils, reducers, sagas, containers, components.
% TODO: Note that some dependencies are eliminated via dependency injection in
  % redux-architecture (+ example!)
% TODO: Improve diagram, use a standard notation (see slides from Software Engineering)
% TODO: Explain the difference between component and containers.

%TODO: redux-architecture (data+events flow) diagram (specifically for our app) (and show how the flow of events is easier to reason about in React+Redux (than in Angular)).

%% TODO: at least one sentence about React components?
%\subsection{React Components}
%
%\begin{itemize}
%\item mostly declarative - simple mental model: rebuilding from scratch every time anything change -> less error prone
%\item reusability -> use of single component on many places in different contexts,
%  or even outside the app (use space-world for ai-search-workshop)
%\item example from our codebase (code, image)
%\end{itemize}

% TODO: This might be relevant if shown for code interpretation instead of
% sending feedback.
%\subsection{Asynchronous Side Effects}
%\label{sec:robomission-asynchronous-side-effects}
%
%Frontend applications are usually full of asynchronous side effects
%(e.g. fetching data from server, wating for user actions).
%Many ways to handle them were proposed,
%such as callbacks (REF) or promises (REF).
%%The most basic one are callbacks --
%%asychronous function takes a function (``callback'') as a parameter
%%and calls it once the asynchronous action is resolved.
%%(TODO: mention/explain promises -- advantage: very explicit; clean error
%%handling; show example for data fetching)
%
%However, both callbacks and promises become awkward for expressing complex
%asynchronous flows, such as visualizing code execution,
%leading to unreadable ``callback hell''. % TODO: ref for callback hell
%Sagas provide an alternative way of handling asynchronous effects using generators.
%Instead of performing asynchronous effects directly, sagas yield
%descriptions of such effects.
%As an example, there is a saga responsible for processing
%submitted feedback.
%Note that while the code contains many asynchronous effects,
%it can be read nearly as easy as standard linear synchronous code.
%
%% TODO: insert comments in the code
%% TODO: mention other advantage of sagas - great testability
%% TODO: also mention new async-await concept
%
%\begin{lstlisting}[language=ES6]
%// Generator for a single submit-feedback request
%function* submitFeedback(action) {
%  const { text, email } = action.payload;
%  // Asynchronous request to get a value from state
%  const url = yield select(getFeedbackUrl);
%  try {
%    // Asynchronous request to post data to server
%    yield call(api.sendFeedback, url, text, email);
%    // Asynchronous request to dispatch a new action
%    yield put(actions.submitFeedback.success());
%  }
%  catch (error) {
%    const { fieldErrors } = error;
%    yield put(
%      actions.submitFeedback.failure(fieldErrors));
%  }
%}
%\end{lstlisting}

% TODO: add code samples for each concept (react component + image, reducer, saga)
% TODO(optional): awesome ES6 (example from our code)
% TODO(Material Design): example of our component + code

% \section{Backend}

%\begin{itemize}
%\item Django, Django Rest Framework
%\item django apps (python packages): learn, monitoring (potential diagram for planned architecture: users, tasks, learn, adaptability, monitoring, simulations, analysis)
%\item models (...), serializers, views, services/use cases/core
%\item data export
%\item monitoring app, metrics computation
%\item generators (e.g. metric computation)
%\item architecture (of individual apps): viewsets/management, services,
%serializers, models, core (actions, credits, recommendation)
%\end{itemize}

Backend is decomposed into modules defining database entities,
their serializers (JSON for sending to frontend and CSV for exports),
\emph{ViewSets} (CITE) describing REST API,
and core modules with pure functions
for computing performance, skills, and recommendation.
% TODO: + domain parsing, monitoring.
%Technologies are becoming obsolete soon,
%so will not discuss them in the text.
An overview of technologies used currently in the project
is included in \cref{chap:technologies}.

\section{Domain Representation}

Domain is represented by a JSON file describing all problem sets %missions, phases, and
and relationships between them, as well as their setting (e.g. toolbox to use)
and name of tasks they contain.
Each task is then described in its own file (\cref{sec:impl.task-sources}).
Furthermore, there is a separate JSON file containing values for parameters
of domain entities (e.g. a time threshold for good performance for each task).
It is convenient to have the relationships and the parameters in different files,
because the former is currently edited by people, while the latter is
a result of a computation.

\subsection{Task Sources}
\label{sec:impl.task-sources}

Each task is described by a single file in a markdown-based format (CITE),
containing its name, setting, and solution.
The high-level grammar for task descreption and an example is shown in
\cref{fig:task-source}.

% There is also top-level [- option: value]*, currently not used
\begin{figure}[htb]
\centering
\begin{subfigure}{.49\textwidth}
{\lstset{numbers=none, showlines=true}
\begin{lstlisting}
# <name>

## Setting
```
<SpaceWorld>
```

[- option: value]*


## Solution
```
<RoboCode>
```


\end{lstlisting}}
\end{subfigure}
\begin{subfigure}{.49\textwidth}
{\lstset{numbers=none}
\begin{lstlisting}
# turning-left

## Setting
```
|bM|b |b |bM|b |
|kA|k |kM|k |kA|
|k |k |kA|kM|k |
|kM|k |kS|k |kA|
```

## Solution
```
left()
fly()
fly()
```
\end{lstlisting}}
\end{subfigure}
\caption{%
  Left: High-level task source grammar.
  Right: An example of a task source
  (task Turning Left, rendered in \cref{fig:robomission-task1}).}
\label{fig:task-source}
\end{figure}

SpaceWorld and RoboCode fragments must follow their own grammars, which
are described in \cref{sec:impl.spaceworld,sec:robocode}.
Currently, there are only two setting options: length and energy limits.
%Currently, there is one mandatory top-level option (task category)
%and two optional setting options (length, energy).
%(TBA: ref to example below; ref to 2 subparts - SpaceWorld grammar, RoboCode grammar)
% [consider] Markdown files are then parsed and loaded into DB by a single command.
% [consider]
% - Localized task names are not part of the task source,
% because all localized messages live on the same place in a single file on FE.
% - PEG grammar for parsing
% - internal represenation (DB model and its serializers for FE and for CSV exports)


Task sources in markdown files have several advantages:
% as opposed to have them in DB
they are human readable,  % + rendered by GH to a nicer presentation
each change is version-controlled,
and the task can be edited easily in any text editor.
However, it is still more convenient to edit tasks in the online game editor
(\cref{sec:robomission.task-editor}),
unless it is a minor modification,
or unless one wants to change multiple tasks at once by a batch command.

\subsection{SpaceWorld Description}
\label{sec:impl.spaceworld}

% TODO: Unify "game world" -- "space world"
Each SpaceWorld (\cref{sec:robomission.game-world}) is described by a simple
human-readable string.
See figure \ref{fig:spaceworld-in-editor} for an example and descripton.
Set of valid SpaceWorld descriptions can be described by the
following context-free grammar:

\begin{lstlisting}
SpaceWorld -> Row+
Row -> '|'(Field'|')+ EOL  // EOL = End Of Line
Field -> Background Object*
Background -> 'r' | 'g' | 'b' | 'y' | 'k'
Object -> 'S' | 'A' | 'M' | 'D' | 'W'
\end{lstlisting}

% TODO: PEG grammars for implementation (why)


%\begin{figure}[h]
%\begin{center}
%\begin{subfigure}{.4\textwidth}
%\centering
%\includegraphics[width=.9\textwidth]{img/spaceworld}
%\end{subfigure}
%\begin{subfigure}{.36\textwidth}
%\centering
%{\lstset{numbers=none}
%\begin{lstlisting}
%|b |b |b |b |b |b |
%|kA|kA|kA|kA|kA|kM|
%|k |k |kW|k |k |k |
%|k |y |k |k |k |r |
%|kM|kM|kA|k |y |k |
%|k |k |kA|k |k |k |
%|k |r |kA|y |k |k |
%|y |k |kA|k |k |k |
%|kS|k |k |kW|k |k |
%\end{lstlisting}}
%\end{subfigure}
%\end{center}
%\caption{Example of Space World with its source code. TODO: Describe letters; consider replacing code listing with a screenshot from task editor with code highlighting}
%\label{fig:spaceworld-source}
%\end{figure}


\imgW[0.7]{spaceworld-in-editor}%
{Example SpaceWorld with its description. % source code.
Each line represents one row of the grid
and is split by pipes (``|'') into fields.
Each field starts with a lower-cased letter denoting color of the field
(e.g. b = blue, k = black),
followed by an optional upper-case letter denoting an object
(A = asteroid, S = spaceship, etc.).
For example, ``rD'' is a red field with a diamond.}



\section{RoboCode}
\label{sec:robocode}

\emph{RoboCode} is a language based on Python,
that is used for representing solutions in task sources.
It closely corresponds to the text written on Blockly blocks;
with just a few exceptions, such as added parentheses
(e.g. ``fly()'' instead of ``fly'')
and shortened literals (e.g. ``b'' instead of ``blue'').
% difference: parenthesis (color() vs. color) and literals ('b' vs. blue)
% TODO: check the statement above
It is also intended to be used in more advanced problem sets as the
transitional phase from block-based to text-based programming.
The language was designed in the following requirements in mind:
simplicity for beginners, understandable even without its previous knowledge,
conciseness (short, but readable programs),
and matching Python closely (for easy transition to Python).

%\subsection{Syntax and Semantic}
%\label{sec:syntax-semantic}

% TODO[consider] note on mixing lexer+parser+some semantic analysis
% (which is partially convenient, partially confusing)

There are four basic commands: % for performing actions:
\texttt{fly()},
\texttt{left()},
\texttt{right()}, and
\texttt{shoot()}.
%\begin{lstlisting}
%fly()
%left()
%right()
%shoot()
%\end{lstlisting}
Each action is combined with moving one row forward.
The movement takes place after the action, with the exception of left and right turning actions, where the movement and the action happen simultaneously,
i.e. the spaceship flies diagonally to the left or to the right.
% TODO: Rephrase. Is confusing, because the general statement ("movement takes
% place after the action") is only relevant for shoot(), and all the other
% commands are special cases.

Loops and conditional statements are same as in the Python,
with the exception of the repeat loop,
which was simplified to a form
% matching a corresponding Blockly block, which is
easier to understand by beginners:
% TODO: ass robocode highlighting
\begin{lstlisting}
repeat 4:
    fly()
\end{lstlisting}
Tests inside while-loops and if-statements are limited to the following forms:
% (again in order to match the respective Blockly blocks):
\begin{lstlisting}
position() [==|!=|>|<|>=|<=] [1..6]
color() [==|!=] ['r'|'g'|'b'|'y'|'k']
<test> [and|or] <test>
\end{lstlisting}
The letters in the color-test have the same meaning as in the space world description,
('r' for red, 'g' for green, etc.)
REF example below


\begin{figure}[h]
\begin{lstlisting}
while color() != 'b':
    if position() == 1:
        right()
    if position() >= 4:
        shoot()
    fly()
\end{lstlisting}
\caption{Example of a complete RoboCode program}
\label{fig:robocode-example}
\end{figure}


\subsection{RoboAST}

\begin{table}[h]
\begin{center}
\begin{tabular}{l l l}
\toprule
Name & Form & Usage  \\
\midrule
RoboCode     & textual (Python-like) & sample solutions in task sources  \\
MiniRoboCode & compact string & logging, storing in DB, analysis  \\
RoboBlocks   & Blockly blocks & code editor for students  \\
RoboJS       & textual (JavaScript) & interpretation in browser  \\
RoboAST      & json (AST) & common intermediate representation \\
\bottomrule
\end{tabular}
\end{center}
\caption{Different code representaions used within the system.}
\label{tbl:code-representation}
\end{table}

While Python-like RoboCode is convenient for writing sample solutions,
more comact form would be useful for logging, storing in DB and analysis.
Furthermore, we want a visual blocks presentation of the code for students.
Last but not least, a JavaScript equivalent of the code is needed for
interpreting the code in the browser.
(Table \ref{tbl:code-representation} shows overview of all representations.)

To avoid implementing separate transformations between each pair of these
presentations, we introduced a common intermediate representation,
\emph{RoboAST}, which is a simple Abstract Syntaxt Tree in json.
(See example of a RoboAST in figure \ref{fig:robocode-transformations-example}.)
For each supported presenation it is enough to implement
its parser returning RoboAST object, and
its generator from RoboAST (figure \ref{fig:robocode-transformations}).
With one parser and one generator for each presentation,
it is possible to transform presentation A
into presentaion B by parsing A into RoboAST and
then generating B from this RoboAST.
This mechanism also allows for switching between RoboCode and RoboBlocks
anytime during writing a solution in the task editor.
% TODO: mention the disadvantage - not able to use functionality provided by Blockly (code generators)

\imgW[0.8]{robocode-transformations}{Transformations between RoboAST and four possible code presentations -- RoboBlocks, RoboCode, MiniRoboCode, and RoboJS.}

\imgW{robocode-transformations-example}{Example of a RoboAST with corresponding RoboBlocks, RoboCode, MiniRoboCode, and RoboJS.}


\subsection{Parsing expression grammar}

For parsing RoboCode into RoboAST, we use a \emph{parsing expression grammar}
(PEG) (REF: PEG grammar).
% TODO: verify the following statement
PEG grammars are basically context-free grammmars with ordered rules
and lookahead expressions (TODO: example).
The specific implementation we used (REF) allows for specifying how should each
parsed subexpression be transformed into a subtree of the final AST
The specyfication can be an arbitrary JavaScript code returning the subtree.
% TODO: check grammar
For illustration, there are some examples of RoboCode parsing rules:

\begin{lstlisting}
CompoundStatement
  = IfStatement
  / WhileStatement
  / RepeatStatement

WhileStatement
  = "while" __ t:Test ":" b:Body
    { return { head: "while", test: t, body: b } }

Test
  = CompoundTest
  / SimpleTest
\end{lstlisting}

The main advantage is that it does not require any lexer and the resulting grammar
is still simple and readable.
The PEG grammar for RoboCode (as described in section \ref{sec:syntax-semantic})
has about 100 lines of code.
However, some preprocessing of the RoboCode is needed, because
PEG is a context-free grammar,
while RoboCode is a context-sensitive language
-- the context is created by indentation.
In addition, we also want to store line numbers alongside the statements
(useful for meta-interpreting:
showing executed line,
linking errors to the point in source code).
Therefore, the preprocessing step transform the code in a context-free form,
where each line of the code is prepended corresponding line number,
and adding and removing indentation levels is denoted by ``>'' and ``<'' characters respectively.
For example, the preprocessed code for the program shown in figure
\ref{fig:robocode-transformations-example} would be:

\begin{lstlisting}
1| shoot()
2| repeat 4:
>
3| right()
4| left()
<
\end{lstlisting}

The figure \ref{fig:robocode-transformations-example} also shows the final RoboAST object.

\subsection{MiniRoboCode}

MiniRoboCode is a condensed form of the RoboCode
which replaces indentation with curly brackets,
key words and functions by their first letters,
and removes any other whitespaces
in order to fit programs into a single short string.
The mapping from the RoboCode can be described by the rules
shown in figure \ref{fig:minirobocode-transformation-rules}.

\begin{figure}[h]
\begin{subfigure}{.49\textwidth}
{\lstset{numbers=none}
\begin{lstlisting}
repeat --> R
while --> W
if --> I
else --> /
position() --> x
== --> =
\end{lstlisting}}
\end{subfigure}
\begin{subfigure}{.49\textwidth}
{\lstset{numbers=none}
\begin{lstlisting}
fly() --> f
left() --> l
right() --> r
shoot() --> s
color == 'y' --> y
color != 'y' --> !y
\end{lstlisting}}
\end{subfigure}
\caption{Transformation rules from RoboCode to MiniRoboCode. (In reality, the transformations goes through RoboAST.)}
\label{fig:minirobocode-transformation-rules}
\end{figure}

Figure \ref{fig:minirobocode-transformations} shows
two examples of complete transformations.
% TODO: use the fancy arrow-split component from my bachelor thesis
\begin{figure}[h]
\begin{subfigure}{.49\textwidth}
{\lstset{numbers=none}
\begin{lstlisting}
fly()
while color() == 'b':
    left()
    right()
fly()
-------------------
fWb{lr}f
\end{lstlisting}}
\end{subfigure}
\begin{subfigure}{.49\textwidth}
{\lstset{numbers=none}
\begin{lstlisting}
repeat 6:
    if position() > 1:
        shoot()
    else:
        right()
-------------------
R6{Ix>1{s}/{r}}
\end{lstlisting}}
\end{subfigure}
\caption{Examples of transformations of RoboCode to MiniRoboCode.}
\label{fig:minirobocode-transformations}
\end{figure}

MiniRoboCode is useful not only for logging and storing programs in DB,
but also for many analyses,
because it is easy to process them by counting letters or matching simple regular expressions,
and because they are short enough to be used as labels in plots.


\subsection{RoboBlocks}

Blockly (REF: blockly) is an implementation of block-based programming
environment from Google,
which we use in the code editor for students
(shown e.g. in figure \ref{fig:robomission-task1}).
% TODO: ref to section about block-base programming envs and their pros/cons
% TODO: mention advnatages of using Blockly from Google: well-tested, widely-used
% (and disadvantage: don't play nicely with modern JS workflow; sometimes needs
% awful hacks to digging in the source code to achieve a desiered behavior
Blockly allows to import and export the currently assembled program in
an XML format, that we call \emph{RoboBlocksXML}.
(See figure \ref{fig:roboblocks-xml} for an example of the XML.)
Both transformations between RoboBlocksXML and RoboAST are straightforward.

% TODO(opt): some details about Blocly? (definition of blocks, localization)

\begin{figure}[h]
\begin{lstlisting}
<xml xmlns="http://www.w3.org/1999/xhtml">
 <block type="start">
  <next>
   <block type="shoot">
    <next>
     <block type="repeat">
      <field name="count">4</field>
      <statement name="body">
       <block type="fly">
        <field name="direction">right</field>
        <next>
         <block type="fly">
          <field name="direction">left</field>
         </block>
        </next>
       </block>
      </statement>
     </block>
    </next>
   </block>
  </next>
 </block>
</xml>
\end{lstlisting}
  \caption{%
    RoboBlocksXML for the program shown in figure %
    \ref{fig:robocode-transformations-example}.}
\label{fig:roboblocks-xml}
\end{figure}

As the RoboBlocks are used by children, it is important to use localized labels
on the blocks.
Depending on the language domain, we initialize Blockly blocks with the corresponding
version of localized block labels.
(See figure \ref{fig:roboblocks-english-czech} for an example of the same
program in two languages.)
Blockly supports its own localization mechanism that fits well
within the localization framework used in our project.

\imgW[0.6]{roboblocks-english-czech}{Same program in English and Czech localizations of RoboBlocks.}
% TODO: Fix the image so that both images are of the exectly same size and sharpness.

\subsection{RoboJS}
\label{sec:robomission-robojs}

For the interpretation of the student code we need yet another representation,
which is a JavaScipt that can be fed into the JS-interpreter.
(See section \ref{sec:robomission-interpretation} for the details of the interpretation).
Figure \ref{fig:robojs-example} shows an example of a generated RoboJS.

\begin{figure}[h]
\begin{subfigure}{.36\textwidth}
\centering\includegraphics[width=.8\textwidth]{img/roboblocks-english}
\end{subfigure}
\begin{subfigure}{.62\textwidth}
{\lstset{numbers=none}
\begin{lstlisting}
highlightBlock(1);
shoot();
highlightBlock(2);
for (var i1_=0; i1_<4; i1_++) {
  highlightBlock(3);
  right();
  highlightBlock(4);
  left();
}
\end{lstlisting}}
\end{subfigure}
\caption{%
  Examples of a RoboJS (right) for a given RoboBlocks program (left). %
  Repeat loops require to generate unique names for iteration variables.
  Each original command is complemented by \texttt{highlightBlock(blockId)},
  so that the meta-interpreter knows which block to highlight.}
\label{fig:robojs-example}
\end{figure}

The generated JavaScript assumes to be executed within the context providing hooks
for actions (\texttt{fly}, \texttt{left}, \texttt{right}, \texttt{shoot}),
sensor functions (\texttt{color}, \texttt{position}),
and for meta-interpreting (\texttt{highlightBlock}).

We have not implemented a parser from RoboJS to RoboAST,
simply because we do not need this tranformation;
otherwise, there is no technical obstacle and the parser would be based on a similar PEG gramamar as for the RoboCode.


\subsection{Interpretation}
\label{sec:robomission-interpretation}

Instead of implementing our own interpreter of RoboCode,
we first transform the code through RoboAST into RoboJS
(described in section \ref{sec:robomission-robojs})
and then use JS-interpeter (REF) to run the code.
We could use any other interpreter that would satisfy the following
conditions:
\begin{enumerate}
\item is implemented in JavaScript (in order to run in the browser),
\item interprets a language that can be generated easily from the RoboAST
  (which includes most languages),
\item allows to define custom asynchronous hooks
  (needed for actions, sensors, and block highligting),
\item allows to step the code, or at least to impose a step or time limit
  (in order to break infinite loops).
\end{enumerate}
% TODO(opt): why not to use JS interpreter in browser: safety (sandbox requirement)
% + probably also not able to step the code or set the time limit, but I haven't verified this
There is nothing special about using JavaScript as the language being interpreted,
other that the JS-interpreter satisfies all our requirements
and is relatively simple to use.

The interpreter itself is a generator that yields actions, sensor requests and
meta-interpreting effects (block highlighting, checking for interruption).
The generator is wrapped into a saga
(see section \ref{sec:robomission-asynchronous-side-effects})
that handles all these asynchronous effects.

\chapter{Analysis of Collected Data}
\label{chap:analysis}

\itquote[0mm]{
The goal is to turn data into information, and information into insight.
}{C. Fiorina}

In order to get insight into how the deployed system works
and what should we focus on in next iterations,
we analyze collected data.
All analyses in this chapter use data collected  % avoid repeating "collected"
during four months (10th November 2017 -- 9th March 2018).
This data, as well as the code performing the analyses, are available
as attachments of this thesis
(described in \cref{sec:attachment.collected-data,sec:attachment.analyses}).


\section{Data Description}

During the four months, about 1000 users tackled at least one task,
%\footnote{The true number of unique users might be lower, because a single user
%may be counted multiple times if he had not signed up and used the system again
%after a session cookie had expired. However, the Google Analytics suggests
%that the number is approximately correct.},
and 800 of them solved at least one task.
About 100 students returned and solved another task another day.
\Cref{fig:engagement-curves} shows complete engagement curves.
% TODO (GA):
%Nearly all users access the webpage on a desktop computer ($94\%$).
% TODO: and does any of the mobile-users solved any task? (because that would be very difficult)
% Desktop 94%, Mobile 4%, Tablet 2%

\begin{figure}[htb]
\centering
\begin{subfigure}{.49\textwidth}
\centering
\includegraphics[height=42mm]{img/engagement-tasks}
%\caption{How many students solved given number of tasks.}
\end{subfigure}
\begin{subfigure}{.49\textwidth}
\centering
\includegraphics[height=42mm]{img/engagement-days}
%\caption{How many students solved a task given number of days.}
\end{subfigure}
\caption{%
  Engagement curves show how many students solved at least given number of tasks (left),
  and how many students solved a task at least given number of days (right).}
\label{fig:engagement-curves}
\end{figure}


About 11000 tasks were attempted (counting only attempts with at least one edit),
and 9600 of them were solved,
resulting in the overall success rate $86\%$.
The daily numbers of solved task sessions are not stable,
with high peaks on days when RoboMission was used
in a programming competition or Hour of Code session at a school
(\cref{fig:solved-count}).
% The following can be a separate paragraph, if another
% analysis is performed.
Over 180 thousand program snapshots was collected,
of which 140 thousand corresponds to edits
and 40 thousand to executions.

% TODO: join with another plot (-> two plots on a single line).
% (e.g., a metric mentioned at the theory part)
% TODO: crop/tighten margings (esp. the bottom one)
%\imgW[0.3]{daily-task-sessions}{%
%  Daily number of solved task sessions (weekly averaged).}
\begin{figure}[htb]
\centering
\begin{subfigure}{.49\textwidth}
\centering
\includegraphics[height=39mm]{img/daily-task-sessions-cropped}
\caption{Daily number of solved tasks\\(weekly averaged).}
\label{fig:solved-count}
\end{subfigure}
\begin{subfigure}{.49\textwidth}
\centering
\includegraphics[height=39mm]{img/task-sessions-time-log-cropped}
\caption{Distribution of log-transformed solving times.}
\label{fig:solving-times-all}
\end{subfigure}
\caption{Task sessions.}
\label{fig:daily-task-sessions}
\end{figure}

Median solving time is 1 minute (interquartile range: 24--145 seconds).
Solving times follow log-normal distribution (\cref{fig:solving-times-all}),
so the mean solving time is much higher
(about 3 minutes). %195 seconds, % with high standard deviation of nearly 500 seconds.
% even if the times are capped at one hour).
% and not very informative


\section{System Behavior}

We can use the collected data to check the requirements on system behavior mentioned in
\cref{sec:robomission.behavior}. Unfortunately, the system is currently not logging
which tasks were recommended and which were self-selected, so we cannot determine
how much are the results influenced by the adaptive behavior. % of the system.
From all attempts, 86\% are successful, and 84\% are solved in less than 15 minutes.
If we look at the first 5 practiced tasks for each student, 87\% of them are solved,
85\% are solved in less than 15 minutes, but only 66\% are solved in less than 2 minutes.

% TODO: how to create horizontal space between them?
\begin{figure}[htb]
\centering
\begin{subfigure}{.48\textwidth}
\centering
\includegraphics[height=40mm]{img/median-time-order-cropped}
\caption{Median solving time and average level per task order during practice.
  First 10 tasks have the median solving time below 1 minute.}
  % The median is below one minute for first 10 tasks.}
\label{fig:solving-times-per-order}
\end{subfigure}
\begin{subfigure}{.50\textwidth}
\centering
\includegraphics[height=40mm]{img/levels-time-cropped}
\caption{Distributions of log-transformed solving times for all levels,\\
         with highlighted medians and interquartile ranges.}
\label{fig:levels-time}
\end{subfigure}
\caption{Levels and solving times.}
\label{fig:solving-times}
\end{figure}


The median solving time of the first 10 practiced tasks is below 1 minute.
Surprisingly, the median time does not increase too much even for the later
attempts during practice (\cref{fig:solving-times-per-order}).
The curve of median times stays similar
if only students with at least 40 task sessions are included in the computation,
which rules out the possibility that this behavior is caused by \emph{attrition bias}
(only better students staying in the system, making the later tasks seem easier than they are).
The plot of the average level per task order during practice
(\cref{fig:solving-times-per-order} bottom) reveals a possible explanation:
an average student is still in the 4th level after 30 tasks,
and the 3rd and 4th level have similar difficulty with a median solving time below 90 seconds
(\cref{fig:levels-time}).
% + random recommendation of tasks from lower level + learning



After 10 minutes of practice, 73\% of students have reached the second level,
and after 20 minutes, nearly 70\% of students have started practicing loops.
Then the progress through levels slows down, and only 37\% of students
are practicing both loops and conditional statements after the first
hour of the tutorial (\cref{fig:students-at-levels}).

\begin{figure}[htb]
\centering
\begin{subfigure}{.49\textwidth}
\centering
\includegraphics[height=42mm]{img/students-at-levels}
%\caption{How many students solved given number of tasks.}
\end{subfigure}
\begin{subfigure}{.49\textwidth}
\centering
\includegraphics[height=42mm]{img/task-sessions-at-levels}
%\caption{How many students solved a task given number of days.}
\end{subfigure}
\caption{%
  Left: the proportion of students in the system after some time of practicing,
  who progressed to the 2nd, 3rd, and 6th level (sequence of commands, repeat loops, if-statements).
  Right: how the proportion of tasks at different levels is shifting during the practice.}
\label{fig:students-at-levels}
\end{figure}

\section{Performance Measurement}

% In ... we have discussed importance of a performance compression
Measurement of performance has an enormous impact on the estimated skills, %in the student model,
and transitively on %the behavior of the tutor model. %, e.g.,
the task recommendations. % to the student.
Currently, we measure %three-valued performance representation, and a
performance using only observed solving time
(\cref{sec:robomission.student}).
However, due to variations in speeds of different students
(which are influenced by external factors, such as used device),
solving time is a rather noisy approximation of the true performance.
Could using other observational data, such as the number of edits,
bring more information about the performance?

%(TODO: Goal: find a compression function that maximizes performance of student model.)
Overall Pearson correlation between solving times and the number of edits (both
log-transformed) is quite high, about 0.73 (\cref{fig:performance-corrs-ts}).
However, if computed for individual tasks,
median correlation drops to 0.64, % (interquartile range 0.5 -- 0.72),
and $25\%$ of tasks have correlation below 0.5 % check the number
(\cref{fig:time-vs-edits}), which suggests that incorporating the number of edits into the
measurement should be considered.

We propose the following heuristic for combining solving times with the number
of edits and runs (executions): a log-transformed sum of edits, runs, and \emph{thinking
actions}.
Thinking action is only assumed to be between two edits or runs if the time interval
is at least 5 seconds long, in order to reduce the noise caused, e.g., by
different screen sizes
(5 seconds seems to be enough to drag and drop a block anywhere on any screen).
If the interval is very long,
we count it as multiple thinking actions (taking $log_5$ from the time interval
and rounding it down to the closest integer).

With this definition, an average student performs about 5 thinking actions
in the first two levels (practicing only sequences of commands),
and about 10 thinking actions in the 3rd and 4th level (practicing repeat and while loops)
(\cref{fig:actions-thinkings}).
This heuristic retains reasonable correlation with both solving time
and number of edits for all tasks (\cref{fig:performance-corrs-p10}).

% the same with thresholds derived from task median (computing 3-level performance)
% -> measure agreement on all task sessions (not correlation, because specific
%  values are now important; prabably RMSE to penalize larger discrepances severaly).
% TODO: 3rd level of abstraction:
% - explore influence of different aggregations, e.g. all-mean (default), min-per-task, median-per-task,
% - explore the role of thresholds
% - explore influence of error metrics; of number of levels
% - explore usage in predictive models on the next performance (independent/cross)
% - explore the action-count measure (with thresholds): e.g., how it behaves wrt. incrasing levels


\begin{figure}[htb]
\centering
%\captionsetup[subfigure]{width=0.90\textwidth,justification=raggedright}%
\begin{subfigure}{.49\textwidth}
\centering
\includegraphics[height=49mm,trim={34mm 0 11mm 0},clip]{img/performance-corr-ts}
\caption{Pearson correlation between performance measures computed\\over all task sessions.}
\label{fig:performance-corrs-ts}
\end{subfigure}
%\begin{subfigure}{.33\textwidth}
%\centering
%\includegraphics[height=32mm,trim={34mm 0 0 0},clip]{img/performance-corr-tasks-med}
%\caption{B.}
%\end{subfigure}
\begin{subfigure}{.49\textwidth}
\centering
\includegraphics[height=49mm,trim={34mm 0 11mm 0},clip]{img/performance-corr-tasks-q10}
\caption{Aggregated per task, bottom 10th percentile
  (i.e., for 10\% of tasks the correlation is below the shown value).}
\label{fig:performance-corrs-p10}
\end{subfigure}
\caption{Correlation between four performance measures
(all of them log-transformed):
E = number of edits, R = number of runs, S = sum of edits, runs, and thinking actions,
T = solving time.}
\label{fig:performance-corrs}
\end{figure}


\begin{figure}[htb]
\centering
\begin{subfigure}{.49\textwidth}
\centering
\includegraphics[height=38mm]{img/time-edits-corr-cropped}
\caption{
  Distribution of Pearson correlations between median solving times and number
  of edits (both log-transformed) for individual tasks.}
\label{fig:time-vs-edits}
\end{subfigure}
\begin{subfigure}{.49\textwidth}
\centering
\includegraphics[height=39mm]{img/actions-thinkings-cropped}
\caption{Median number of actions (edits, runs, and thinking actions)
  with the increasing level.\\~}
\label{fig:actions-thinkings}
% TODO: add interquartile ranges
\end{subfigure}
\caption{Analysis of performance measures.}
\end{figure}


\begin{figure}[htb]
\centering
\includegraphics[width=\textwidth]{img/difficulties-tasks-levels}
\caption{%
  Difficulties of tasks in the first 6 levels.
  The difficulty is measured as success rate and median solving time.}
\label{fig:difficulties-tasks-levels}
\end{figure}




\section{Task Difficulties}

% Reformulate to make it clear, that it is the analysis before introduction of the phases,
% that actually leads to the refinement described in the thesis.

% TODO: Resolve mission vs. level.
Our tutor model (\cref{robomission.tutor}) assumes that difficulties of tasks
in each phase are approximately the same,
and that the overall difficulty is increasing as the level increases.   % even phase.
Using collected data, we can determine whether these assumptions are satisfied,
and if not, suggest adjustments to improve their validity.
\Cref{fig:levels-time} shows that on average the difficulty of levels
is increasing, but it also reveals that level 7 is more difficult than level 8.
Levels 3 and 4 have similar difficulty, which is expected because they practice
two similar concepts, repeat and while loops.

% TODO: Elaborate on the solution to this issue.
% (TODO?: phases instead of missions, and select only a few to save space)
Looking at the difficulty of individual tasks (\cref{fig:difficulties-tasks-levels})
help us to discover outliers, whose difficulty is significantly
different from the other tasks in the same level.
The \cref{fig:difficulties-tasks-levels} also suggests that dividing levels
into phases is necessary to achieve reasonable homogeneity. For example, in the 2nd
level (World), there are two clear groups of tasks with significantly different
difficulty. Although in the other levels such a clear split does not exist,
they still contain tasks with a wide range of difficulties. %, so the refinement into phases seems appropriate.
% TODO: note/analyze robustness of these analyses (given the limited data)

%\imgW{difficulties-tasks-levels}{%
%  Difficulties (spent times and number of edits, both log-transformed)
%  of tasks in each level.}

\chapter{Conclusion}
\label{chap:conclusion}

\section{Summary}
\label{sec:conclusion.summary}

TODO:
- AI for estimating skills and task difficulties -> task recommendation -> flow
  (more detailed: domain, student, tutor model, analysis, ...)
- developped ALS for introductory programming RoboMission, HoC mode
  (already testes in schools) + in competitions (Purkiada, Sob 2017, Sob 2018)
% Sob - consider including a photo

\section{Future work}
\label{sec:conclusion.future-work}

TODO:
- research, Implementation
- see: "Four modes of usage" in chapter 5 + gdoc:alg:future-todo


\printbibliography[heading=bibintoc]

\appendix

%\chapter{Glossary}
%\label{chap:glossary}
%
%\begin{description}
%    \item[Flow] TBA: add definition of flow.
%\end{description}

\chapter{Data Attachment}
\label{chap:data}

TBA: data attachment

\chapter{Used Technologies}
\label{chap:technologies}

Technolgoies used in the project change quite quickly in order to meet new requirements,
especially on frontend, where the tools and best practices are evolving rapidly.
The following table shows the technologies in spring 2018.

%Table \ref{tbl:technologies} shows overview of the technologies used
%in the project in spring 2018.
%Note that they are gradually evolving, either to meet new requirements,
%or simply to replace old technologies with newer ones.
%This is especially true for frontend, where the tools and best practices
%are evolving rapidly.
%When this project started in 2015,
%we used several at that time popular technologies
%(AngularJS, Bootstrap, bower, grunt)
%which became obsolete during next 2 years,
%so we replaced them by newer ones.
%Many technologies were introduced naturally as the projects grew,
%e.g. Django Rest Framework for REST API,
%or redux-saga for cleaner handling of asynchronous side-effects
%(instead of the originally used redux-thunk).

\begin{center}
\begin{tabular}{l l l}
\toprule
Area & Technology & Main Reason  \\
\midrule
\textbf{Project management} & GitHub & easy to use \\  % "Project management"
% Version control & git & easy to use \\
Commands & make, django, npm & easy to use \\
% Monitoring & Google Analytics & widely-used \\
\hline
\textbf{Backend} & Python 3 & concise, high-level \\
%Dependencies & pip & easy to use \\
%Environment & virtualenv[wrapper] & easy to use \\
Unit tests & pytest & concise, readable \\
Web framework & Django & feature-complete \\ %, well-documented \\
REST API & DRF & feature-complete \\ %, well-documented \\
Database & PostgreSQL & widely used \\
Web server & Nginx + Gunicorn & widely used \\
Scheduled jobs & cron, django-crontab & standard tool \\
Data export & Django Rest Pandas & tranformations \\ % before export \\
\hline
\textbf{Frontend} & ES6 & concise, readable \\
Dependencies & npm & easy to use \\
Bundling & webpack & feature-complete \\ %, widely used \\
Compiling & babel & feature-complete \\ % , widely used \\
State & redux  & predictability, testability \\  % behavior, testability \\
Side effects & redux-sagas  & readability, testability \\
Views & React & declarative, reausable \\
Design (UI) & Material-UI & good appearence \\
HTTP client & axios & promises \\
Code blocks & Blockly & well-tested \\
Parsing & PEG JS & declarative rules \\  %(Parsing expression grammar) \\
Interpretation & JS-interpreter & stepping, custom hooks \\
Localization & react-intl & one place for all messages \\
\hline
\textbf{Analysis} & jupyter, pandas, sklearn & interactive, high-level \\
Plotting & matplotlib, seaborn & features; nice plots \\
%Machine learning & sklearn & powerful \\
%\hline
%\textbf{Analysis} & Python 3 & high-level, widely used \\
%Documents & jupyter notebook & interactivity \\
%DataFrames & pandas & feature-complete, well-documented \\
%Plotting & matplotlib, seaborn & powerful, looks good \\
%Machine learning & sklearn & powerful \\
\bottomrule
\end{tabular}
\end{center}


\end{document}
