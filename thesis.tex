\documentclass[
    %twoside,   % Printed version
    %printed,   % Printed version
    digital,    % PC version
    oneside,    % PC version
    color,
    11pt,
    nocover,
    notable,
    nolof,
    nolot,
    final
]{fithesis3}
%% The following section sets up the locales used in the thesis.
\usepackage[resetfonts]{cmap}
\usepackage[T1]{fontenc}
\usepackage[
  main=english, %% By using `czech` or `slovak` as the main locale
                %% instead of `english`, you can typeset the thesis
                %% in either Czech or Slovak, respectively.
  english, czech %german, russian, slovak %% The additional keys allow
]{babel}        %% foreign texts to be typeset as follows:

%%
%% The following section sets up the metadata of the thesis.
\thesissetup{
    date          = \the\year/\the\month/\the\day,
    university    = mu,
    faculty       = fi,
    type          = mgr,
    author        = Tomáš Effenberger,
    gender        = m,
    advisor       = {doc. Mgr. Radek Pelánek, Ph.D},
    title         = {Adaptive System for Learning Programming},
    TeXtitle      = {Adaptive System for Learning Programming},
    keywords      = {adaptive learning, elementary programming},
}


% Setup bibliography.
\usepackage[backend=biber, 		% use biber as backend instead of BiBTeX
	bibstyle=ieee-alphabetic, 	% bibliography style: IEEE with alphabetic citations
	citestyle=alphabetic, 		% citation style
	url=true, 			        % display urls in bibliography
	hyperref=auto,			    % detect hyperref and create links
]{biblatex}
\addbibresource{thesis.bib}


\thesislong{abstract}{%
TBA: abstract
}

\thesislong{thanks}{%
TBA: thanks

TBA: mention RH and MU projects
}

\usepackage{makeidx}      %% The `makeidx` package contains
\makeindex                %% helper commands for index typesetting.



%% These additional packages are used within the document:
\usepackage{paralist}
\usepackage{amsmath}
\usepackage{amsthm}
\usepackage{amsfonts}
\usepackage{url}
\usepackage{menukeys}

% cref, has to be loaded after hyperref
\usepackage{hyperref}
\usepackage[noabbrev,capitalise]{cleveref}
% for long equation wrap
\usepackage{wrapfig}
% figure captions
\usepackage{caption}
% subcaption for 2 figures in one
\usepackage{subcaption}
% H figures
\usepackage{float}
% tables
\usepackage{tabularx}
% colored cells (cellcolor)
\usepackage{colortbl}
% colours
\usepackage{xcolor}
% better typeset of line ends and so (nicer)
\usepackage{microtype}

% package to make bullet list nicer
\usepackage{enumitem}
\setitemize{noitemsep,topsep=3pt,parsep=3pt,partopsep=3pt}

% intendation
\usepackage{parskip}


\begin{document}

\chapter{Introduction}
\label{chap:introduction}

TBA: introduction

\chapter{Learning Programming}
\label{chap:learning-programming}

This chapter describes the current state of the art of teaching programming, both from the view of successful learning systems and from the view of a research on learning programming.

\section{Existing Systems for Learning Programming}
\label{sec:existing-systems}

There are many systems for learning programming.
These systems can be grouped by several criteria:

\begin{itemize}
\item computer systems, or physical toys
\item device: online web pages, mobile apps, offline computer programs
\item target group, e.g. children in a class, children at home, …
\item assumed prerequisites, e.g. reading, writing, computer skills, or even elementary programming
\item covered concepts, e.g. loops, conditions, functions, recursion, particular language features
free or paid
\end{itemize}

In this section, we present the most notable of them with brief descriptions. Next section attempt to extrapolate useful strategies for teaching programming.


\subsection{LightBot}
\label{sec:lightbot}
LightBot%
\footnote{Available at \url{http://lightbot.com/}.}
is a web page with a sequence of tasks for learning programming using block-based programming language.
Students write simple programs to control a robot living in a grid world.
Two types of movement (walking and jumping) enables to create variety of easy tasks.
The system provides clear and simple interactive instructions.
The following concepts are covered: commands, procedures and simple loops via tail-recursion.

// TBA: screenshot


\subsection{Robozzle}
\label{sec:robozzle}
With a robot on a grid  and block-based programming, Robozzle%
\footnote{Available at \url{http://www.robozzle.com/}.}
is similar to LightBot.
However, by adding colors to the grid, which can be used for conditioning and enables difficult tasks with recursion.
Robozzle shows that a carefully chosen set of a few orthogonal commands is enough for creating hundreds of diverse tasks.
Robozzle allows users to create their own tasks.
These tasks can be tried and rated by other learners.

// TBA: screenshot


\subsection{BlocklyGames}
\label{sec:blockly-games}
Blockly is a popular block-based programming interface.
In contrast to the block-based interfaces in LightBot and Robozzle,
Blockly doesn’t constrain the “shape” of the program
(program can be arbitrary long and can have arbitrary many functions).
The main purpose of BlocklyGames%
\footnote{Available at \url{https://blockly-games.appspot.com/}.}
is to provide a demo of Blockly usage.
The webpage consists of several games, ordered in increasing difficulty.
For example, in the first level, students learn how to compose blocks together as a puzzle;
in the second level they learn loops and conditions by solving tasks in a maze;
next they practice compound conditions, etc.
Final level serves as a transition from block-based programming to textual JavaScript programming.
Each game consists of 5-10 tasks, again with fixed ordered by increasing difficulty, with no personalization.
The fixed order enables use of the program from the previous task and thus gradually building more and more complex programs,
resulting in e.g. a fairly sophisticated images in turtle graphics.
Another notable feature of the system are non-ignorable instructions,
which requires to take a described action before they disappear
\cite{blockly-10-things}.

// TBA: screenshot


\subsection{Human Resource Machine}
\label{sec:human-resource-machine}
Human Resource Machine%
\footnote{Available at \url{http://tomorrowcorporation.com/humanresourcemachine}.}
is an example of a paid offline computer game for learning programming using block-based programming interface.
Though being presented as a game,
the player spends nearly all the time solving tasks similar to those as in the previously mentioned learning systems.
The sequence of tasks is non-personalized and nearly linear,
with only a few short side branches.
Main difference from the other systems is a slightly different domain:
it uses more low-level programing commands, such as
input, output, move-to, move-from, add, sub, jump, jump-nonzero, jump-negative.
The game comes with a debugger, giving a possibility to step through the program.
All elements are explained when they first appear and programming blocks can be explained again simply by dragging the block onto a field with a questionmark.

// TBA: screenshot


\subsection{Ozobot}
\label{sec:ozobot}
Ozobot%
\footnote{Information available at \url{http://ozobot.com/}, simulator at \url{http://games.ozoblockly.com}.}
is a small physical robot,
programmable by either a Blockly-based interface or by drawing lines on a paper.
Therefore, it can be used completely without a computer and does not require ability to read and write,
thus being suitable for very young children.

// TBA: screenshot


\subsection{Problem Solving Tutor}
\label{sec:problem-solving-tutor}
Problem Solving Tutor%
\footnote{Available at \url{tutor.fi.muni.cz}.}
includes a few problem sets for practicing programming,
such as Interactive Python, Robot Karel or Robotanist.
Robotanist is a variation on the Robozzle,
offering both easy introductory tasks requiring no previous exposure to programming,
and extremely difficult tasks on sophisticated use of recursion.
After each solved task, Problem Solving Tutor shows a recommendation of two tasks,
one easier and one more difficult,
with a predicted solving time.
According to authors, showing predicted problem solving time to the user serves as a motivational element
– it poses a suitable challenge to ``overcome oneself''
\cite{pelanek-student-modeling-times}.

// TBA: screenshot


\subsection{Project Euler}
\label{sec:project-euler}
The core of Project Euler%
\footnote{Available at \url{projecteuler.net}.}
is a list of several hundreds of programming problems with one correct answer (usually number).
The problems can be solved in any language
and then checked whether the computed answer is correct by the provided answer checker.
The project is not meant to teach elementary programming,
but rather to hone one’s programming skills;
more complicated tasks require even some knowledge of algorithms and data structures.
The system provides several means for motivation:
levels (based purely on number of solved tasks), badges
(e.g. for solving 10 consecutive problems, 50 prime numbered problems, etc.),
comparison with user’s friends,
statistics page with several leaderboard (e.g. by country),
and a special score for solving the most recently published problems.

// TBA: screenshot


\subsection{HackerRank}
\label{sec:hacker-rank}

Online programming contests,
in which people attempt to solve as many problems as possible
in limited time frame of a few hours,
become popular in the last years.
In addition to organizing such contests,
HackerRank%
\footnote{Available at \url{https://www.hackerrank.com/}.}
also provides many training programming tasks for various topics,
from introductory programming to machine learning.
Solutions are evaluated on a server against a prepared set of test cases with time limits.
In addition to the classic motivation in the form of points, badges and leaderboards, students are motivated to practice to perform well in the competitions,
where they can win some prices and sometimes even price offers.
HackerRank also helps student to decide on a problem to solve by showing a difficulty according to the author of the task (on a 3-star scale),
as well as success rate among the past submissions.
Furthermore, after solving a task,
a specific recommendation for one task to solve next is shown.

// TBA: screenshot


\section{Strategies for Easier Learning}
\label{sec:strategies-for-easier-learning}

Learning programming is difficult,
  because it at the same time requires
  to adopt algorithmic thinking,
  understand program execution
  and remember a formal syntax of a programming language.
To make learning easier,
  the systems presented in the previous section use diverse strategies,
  including avoiding syntax errors,
  providing visual output
  and showing helpful hints.
Various other strategies were tried in the past;
paper \emph{Lowering barriers for Novice Programmers}
  \cite{lowering-barriers}
  provides a detailed overview.



\chapter{Template Chapter}
\label{chap:template}

Citation example: \cite{adaptive-practice} and \cite{flow}.


\chapter{Conclusion}
\label{chap:conclusion}

\section{Summary}
\label{sec:conclusion.summary}

TBA

\section{Future work}
\label{sec:conclusion.future-work}

TBA


% ===== APPENDIX AND BIBLIOGRAPHY =====

\printbibliography[heading=bibintoc]

\appendix

\chapter{Glossary}
\label{chap:glossary}

\begin{description}
    \item[Flow] TBA: add definition of flow.
\end{description}

\chapter{Data attachment}
\label{chap:data}

TBA: data attachment

\end{document}
