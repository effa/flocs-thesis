\chapter{Implementation of RoboMission}
\label{chap:implementation-of-robomission}

TBA

\section{System Architecture}

TBA

\section{Frontend}

\begin{itemize}
\item first prototype: Angular + Bootstrap
\item now: ES6, React, Redux, Material Design
\item TBA: explain each of these used technologies (briefly)
\item why the change: more flexible and understandable, easier development and bug fixing; TBA: show how the code is more readable with ES6, how the flow of events is easier to reason about in React+Redux (than in Angular) and how Material Design is more user-friendly than Bootstrap
\end{itemize}

\section{Backend}

TBA

\section{Clean Core}

TBA


\section{SpaceWorld Grammar}

\begin{itemize}
\item TBA
\item together with RoboCode allows for a task sources in human-readable and editable markdown (TODO: insert example of space world and also of full task source with rendered game preview), and convenient online task editor
\end{itemize}


\section{RoboCode}

\begin{itemize}
\item new programming language (RoboCode)
\item transformations (ast <-> RoboCode, MiniRoboCode, Blockly, JS)
\item common roboAST representation serves as a mediator between different languages
      that we need to support (Blockly for kids, RoboCode for writing,
      MiniRoboCode for logging, JS for running)
\item (allows for some cool magic such as immediate bidirecitional translating between blockly and RoboCode)
\item describe RoboCode PEG grammar, iclude same example rules and parse trees
\end{itemize}


