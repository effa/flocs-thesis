\chapter{Attachments}
\label{chap:attachments}

Source code, exported data and performed analyses
are available as attachments in the \emph{Archive of Thesis}%
\footnote{\url{https://is.muni.cz/th/410350/fi\_m/}}
in the Information System of Masaryk University,
and also in GitHub repositories of the project and the thesis.

\section{Source Code}
\label{sec:attachment.source-code}

Snapshot of the source code
(including setup instructions in \texttt{README.md})
% documentation (\texttt{docs}),
%and domain data (\texttt{backend/domain}),
is attached as \texttt{code.zip}.
The most recent version of the project is available at
\url{https://github.com/adaptive-learning/robomission}.

\section{Analyses}
\label{sec:attachment.analyses}

Attachment (\texttt{analysis.zip})
contains several Jupyter Notebooks and Python modules with the analyses
presented in the thesis.
Also available at
\url{https://github.com/effa/flocs-thesis/tree/master/analysis}.

\section{Exported Data}
\label{sec:attachment.collected-data}

Both the domain and collected data used for analyses in this
thesis (exported on 9th March 2018) are attached as
\texttt{data.zip}, and they are also available at
\url{https://github.com/effa/flocs-thesis/tree/master/data}.
Data is exported as a few CSV tables,
with the columns described below.
%Columns of the CSV tables are described in
%\cref{tbl:data.tasks,tbl:data.levels,tbl:data.task-sessions,tbl:data.program-snapshots}.

\begin{table}[htb]
\centering
\caption{Tasks.}
\begin{tabular}{l l}
\toprule
Attribute & Description \\
\midrule
id & ID of the task \\
name & text label of the task \\
level & name of the mission \\
setting & JSON with "fields" and optional limits ("length", "energy") \\
solution & MiniRoboCode representation of the solution (\cref{sec:minirobocode}) \\
\bottomrule
\end{tabular}
\end{table}

% TODO: consider to update to the new data export with PS (of course not only
% description, but also the actual export and analyses)

\begin{table}[htb]
\centering
\caption{Problem sets.}
\begin{tabular}{l l}
\toprule
Attribute & Description \\
\midrule
id & ID of the problem set \\
level & order of the problem set (by difficulty) \\
name & text label of the problem set \\
toolbox & name of the available toolbox \\
tasks & list of task names contained in the problem set \\
\bottomrule
\end{tabular}
\end{table}


\begin{table}[htb]
\centering
\caption{Task sessions.}
\begin{tabular}{l l}
\toprule
Attribute & Description \\
\midrule
id & ID of the task session \\
student & ID of the student \\
task & ID of the task \\
solved & whether the student solved the task (boolean) \\
start & timestamp of opening the task \\
end & timestamp of the last action in the task session \\
time\_spent & length of the task session in seconds \\
\bottomrule
\end{tabular}
\end{table}



\begin{table}[htb]
\centering
\caption{Program snapshots.}
\begin{tabular}{l l}
\toprule
Attribute & Description \\
\midrule
id & ID of the program snapshot \\
task\_session & ID of the task session \\
time & timestamp of creating the snapshot \\
program & MiniRoboCode representation of the code (\cref{sec:minirobocode}) \\
granularity & "edit" or "execution" \\
order & order of the snapshot of this granularity in the session \\
correct & whether the execution was successful (empty for edits) \\
time\_from\_start & seconds from the start of the task session \\
time\_delta & seconds from the previous snapshot of same granularity \\
\bottomrule
\end{tabular}
\end{table}
