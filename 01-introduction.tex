\chapter{Introduction}
\label{chap:introduction}

TODO: general introduction to the topic of AL of introcutory programming

TODO: questions we focus on
- "models": how to model domain and student for introductory programming
- "techniques": how to use AI to improve learning of introductory programming
  in "Hour of Code" tutorials
- "evaluation": how to evaluate the system and its components?
  does the adaptivity improve the learning/engagement?

TODO: goals of the thesis
- answer the questions above
- plus: develop a ALS for IP: why: to better understand the problem,
  to collect data for analyses and test our ideas (support/reject our hypotheses),
  and idally to have a working system that provides the efficent learning experience

TODO: related terms
- flow (...), zone of proximal development (Vygotsky)


TODO: related areas
- introductory programming learning
- adaptive learning / intelligent tutoring systems
- recommendation systems (with performance instead of ratings)
- HCI, (software learnability)
- games/levels design


TODO: structure of the thesis


TODO:
- with different prior knowledge, same assignments in same environment are problematic
\cite{comparing-blocks-text-weintrop2017}:
"This came up a few times in this study, when advanced students lamented
having to use a block-based modality, instead wanting to go straight into learning Java."
