\chapter{Adaptive Learning}
\label{chap:adaptive-learning}

% TODO:
% - incorporate other notes from my/thesis gdoc
% - inspiration from relevant articles
% - check first paragraph using last-year feedback
% - add references for provided examples (playing chess, autonomous car, ...)

Artificial intelligence proved to be a mighty tool
  for tackling difficult algorithmic tasks,
  from playing chess to driving an autonomous car.
The power of artificial intelligence can be also used
  to develop a personalized adaptive system for learning programming.
Such system should create an optimal learning experience for each student
  by providing them with problems of difficulty matching their skill,
  so that the student stays challenged and interested in solving them.

In the existing systems (\ref{sec:existing-systems}),
  the sequence of tasks is the same for everybody.
As a result, the progress is necessarily too slow for some students,
  who could skip some of the tasks,
  while being too fast for others,
  which could highly benefit from solving many more similar tasks.
Artificial intelligence can be used to personalize
  the sequence of tasks for every student.
By giving a student a task of the optimal difficulty
  -- neither too easy, nor too difficult --
  it can help the student to get into a state of flow
  \ref{sec:motivation.challenge}.

In addition to choosing the most suitable task for given student,
  artificial intelligence has also other possible uses in learning systems,
  for example automatic hints generation \cite{generating-hints}
  or skill visualization (TBA: ref).
Furthermore, artificial intelligence techniques can be used
  to analyze collected data offline
  and, for example, detect problematic tasks
  or suggest how to group tasks into categories (TBA: ref).

% TODO: Consider to remove (or add refs)
Adaptive learning systems have been already successful in some domains.
For instance, Map Outlines,
  developed by Adaptive Learning research group at Masaryk University,
  is an intelligent web application for learning geography.
It has been used by tens of thousands of students
  and online experiments have confirmed
  that the adaptivity of the system helps to improve a learning outcome
  (TBA: ref - see gdoc).
In addition to the geography, similar adaptive web applications
  for learning anatomy, biology, elementary mathematics and Czech grammar
  were developed by the research group in recent years
  (TBA: ref/links).

\section{Student Modeling}
\label{sec:student-modeling}

\begin{itemize}
\item TBA
\end{itemize}

\section{Task Recommendation}
\label{sec:task-recommendation}

TBA


\section{Metrics and Evaluation}
\label{sec:metrics-and-evaluation}

TBA


\section{Iterative Improvement}
\label{sec:iterative-improvement}

TBA
