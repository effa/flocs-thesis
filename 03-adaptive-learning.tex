\chapter{Adaptive Learning}
\label{chap:adaptive-learning}

% TODO:
% - incorporate other notes from my/thesis gdoc
% - inspiration from relevant articles
% - add references for provided examples (playing chess, autonomous car, ...)

Artificial intelligence proved to be a mighty tool
  for tackling difficult algorithmic tasks,
  from playing chess to driving an autonomous car.
The power of artificial intelligence can be also used
  to develop a personalized adaptive system for learning programming.
Such system should create an optimal learning experience for each student
  by providing them with problems of difficulty matching their skill,
  so that the student stays challenged and interested in solving them.

In the existing systems (\ref{sec:existing-systems}),
  the sequence of tasks is the same for everybody.
As a result, the progress is necessarily too slow for some students,
  who could skip some of the tasks,
  while being too fast for others,
  who could highly benefit from solving many more similar tasks.
Artificial intelligence can be used to personalize
  the sequence of tasks for every student.
By giving the student a suitable task
  -- neither too easy, nor too difficult --
  it can help the student to get into the state of flow
  (\ref{sec:motivation.challenge}).

In addition to choosing the most suitable task for given student,
  artificial intelligence has also other possible uses in learning systems,
  for example automatic hints generation \cite{generating-hints}
  or skill visualization (TBA: ref).
Furthermore, artificial intelligence techniques can be used
  to analyze collected data offline
  and, for example, detect problematic tasks
  or suggest how to group tasks into categories (TBA: ref).

% TODO: Consider to remove (or add refs)
Adaptive learning systems have been already successful in some domains.
For instance, Map Outlines,
  developed by Adaptive Learning research group at Masaryk University,
  is an intelligent web application for learning geography.
It has been used by tens of thousands of students
  and online experiments have confirmed
  that the adaptivity of the system helps to improve a learning outcome
  \ref{alg.evaluation-geography}.
In addition to geography, similar adaptive web applications
  for learning anatomy, biology, elementary mathematics and Czech grammar
  were developed by the research group in recent years
  (TBA: ref/links).

\section{Student Modeling}
\label{sec:student-modeling}

For the system to be adaptive, models of
  a student skill, task difficulty, and student-task interaction
  need to be designed, implemented and evaluated using collected data.
The purpose of these models is to predict the probability that a given student
  would solve a given task
  and also the time the student would need to complete the task.

Various approaches for student models have been proposed.
For instance, each attempt to solve a task can be interpreted
  as a match between the student and the task and after this match ends,
  skill and difficulty estimates are revised.
If the student solves the task quickly, their skill will be increased
  and the difficulty of the task will be decreased.
On the contrary, if the student fails to solve the task or if takes them too long,
  their skill will be decreased and the difficulty of the task will be increased.
% TODO: this is called elo -> in detail in section x.y, other models a, b, c in sections z, z, z)


\section{Data}
\label{sec:student-modeling.data}

To learn model parameters, such as difficulty of individual tasks, some data is needed.
Different models require different type and amount of data.
Some data about tasks are independent of students and can be used
  for initial task difficulty estimates.
In addition to task statements, sample solutions and expert labels (e.g. levels)
  are sometimes provided by an instructor for each task.

Once the task is deployed in the running system,
  rich data can be collected from each interaction between a student and the task:
\begin{itemize}
  \item whether solved,
  \item solving time,
  \item number of attempts,
  \item number of clicks,
  \item program snapshots,
  \item explicit student' evaluation (or labels). % TODO: needs explanation
\end{itemize}

% TODO: mention that times behave log-normal; and # clicks coveys similar
% information (Klusacek 2014)
% TODO: mention different granularity levels for taking program snapshots

\section{Item Response Theory}
\label{sec:irt}

TODO: explain IRT

%\section{Problem Response Theory}
%\label{sec:problem-response-theory}

Classical IRT was originally developed for knowledge testing
  and so it contains some strong assumptions which does not hold for learning programming.
First, it assumes a single constant skill, and
second, it only deals with correctness (probability of success), not solving times.
However, programming skill seems to be multidimensional;
  for example, one student can be proficient with functions and struggle with loops,
  while another student can master loops and struggle with functions.
Furthermore these skills are changing significantly during interaction with the system,
  at least in ideal case -- because learning (not testing) is the objective.

IRT can be extended to address these issues.
To predict problem solving times instead of probability of success
  \emph{Problem Response Theory} (PRT)
  assumes an exponential relationship between a problem solving skill
  and the time to solve a problem  % TODO: explain + formula
  \cite{alg.problem-response-theory}.
% TODO: Add note that the assumption of exponential relationship is justified
% by observed solving solving times distribution and it is also intuitively
  % plausible - multiplicative nature of solving times.

Similarly, multidiminsional skill and learning can be incorporated
  \cite{pelanek-student-modeling-times}.
% TODO: details and formulas

\section{Elo}
\label{sec:elo}

TODO: explain elo (? from introductory paragraph?) \cite{alg.elo}
Elo can be extended to work with solving times. (TBA: ref)

Advantages of the Elo models are their simplicity, flexibility,
online-computation and reasonably good performance.


\section{Concepts}

% TODO: fix and mention used terminology: concept vs. skill

A lot of research in the area restrict its attention to single-dimensional skill.
This assumption is reasonable for many logic puzzles (e.g. sudoku);
however, programming tasks require diverse skills.
Even on the elementary levels, loops, conditional commands and functions
  can be taught in any order
  and it is perfectly possibly to master one of these skills,
  while struggle with the others, or even not been introduced to them.

As we mentioned in sections \ref{irt} and \ref{elo},
  both IRT and Elo models can be extended to work with multidimensional skills.
% TODO: However, the way they compose multiple skills into a single prediction
% is not completely justified / can differ depending on domain/skills (e.g.
% adding skills vs taking best/worse)

Although modeling multiple skills seems useful,
  there is a classical trade-off between complexity of the model (number of skills)
  and how well (or how fast) can be the parameters estimated.
More parameters require more data for the estimates to converge.
% TODO: which is especially concern for students; 1. predictions needed
% immediately for new students (no/little data), 2. students' skills are
% assumed to change (tasks didn't have these problems)


% \subsection{Learning Concepts from Data}

Concepts can be either defined manually or detected automatically
  \cite{niznan-thesis}.  % TODO: specify chapter/pages
Manually selected concepts, such as loops and conditional commands,
  have the advantage of being interpretable,
  so they can be also use for skills visualizations in the user interface
  to provide students with the information of their learning progress.
Furthermore, no data needs to be collected in advance,
  while the automatic techniques require a lot of data to be stable.  % TODO: how much?


% \subsection{Prerequisites}
Undoubtedly, there are some relationships between skills;
  for example, nested loops cannot be mastered without mastering simple loops.
The hierarchical structure between concepts can be modeled
  as a directed acyclic graph (DAG),
  where each vertex is a concept and each edge represents a prerequisite.
Having DAG of concepts then allows to model students using Bayes networks
  \cite{its-programming}.

% TODO: simple diagram: DAG of concepts example

\section{Task Recommendation}
\label{sec:task-recommendation}

Student models are used by an \emph{instructional policy} to recommend
  the most suitable task for a student.
In spite of having all the predictions about success probabilities
  and time estimates in hand,
  task recommendation is not an easy task.
First, it is not clear what the optimal difficulty even means.
Second, it may vary for different students, domains or types of problems.
Furthermore, optimal difficulty is not the only criterion to be considered.
For instance, diversity of tasks is important to keep students interested.
No principled techniques for task recommendation have been developed yet;
however, several heuristic approaches have been used
  and proved to work well. (TBA: ref)


Note that task recommendation is not a necessary requirement
  for the system to be adaptive.
Student models can be utilized in other means to achieve personalized behavior,
  e.g. mastery learning (section \ref{sec:mastery-learning}).
System can also just provide students with predicted solving time
  or probability of success
  and leave the choice of the next task on them.
Showing predictions can be already perceived as a mild form of recommendation.
Indeed, recommendations can range from \emph{soft} to \emph{hard}.
Soft recommendations can be achieved by
  ordering tasks according to suitability,
  filtering and only showing a subset of tasks,
  or showing suggestion such as
  ``too easy'', ``too difficult'' and ''ready to tackle'' next to each task
  (Similar suggestions in the form of traffic-light colors
  are used in the system described in \cite{its-programming}.)
System can be more strict and show a single recommended task,
  or even enforce the recommendation by immediately progressing student to
  a next task without asking and giving them a chance to select a different task.

% TODO: details about heuristics/methods for selecting single best task

\section{Mastery Learning}
\label{sec:mastery-learning}

TODO: describe mastery learning + usage of student models (to show the progress
and to determine that the student already achieved the mastery)

% TODO: screenshot of a mastery progress bar (e.g. from UmimeX)

\section{Metrics and Evaluation}
\label{sec:metrics-and-evaluation}

To decide if the adaptivity of the system increases the learning outcomes,
  a suitable metric must be chosen and incorporated into the system,
  for example using pre-tests and post-tests.
Ideally, the system should also allow to easily change various conditions
  for subsets of students and hence perform controlled online AB experiments.


\subsection{Metrics for Predictions}

TODO: data we need -- this is simple: we can collect the objective solving times and compare
with predicitons (there is some noise, because of students taking breaks, cheating, etc., but should be ok)

TODO: mention standard metrics and evaluation methodologies in ML (for
predicted times, RMSE vs. MAE etc.)


\subsection{Metrics for Recommendation}

TODO: what data to use? much less clear than  for predictions

TODO: mention standard metrics and evaluation methodologies in RC (for recommended tasks)

The tasks environment in learning systems often differ significantly
  from the real-world environment,
  e.g by using blocks instead of text
  and other aspects mentioned in section \ref{sec:strategies-for-easier-learning}.
This makes evaluation harder, because while the collected data on which we
  evaluate the system comes from the system itself,
  ultimately students leave this simplified environment
  and the performance outside the learning system is the important criterion.

% This issue is mentioned in \cite{challenges-of-blocks-based-environments},
% by Werntrop and Wilensky
% in the context of proper evaluation of block-base programming environments.

% TODO: this relates to next subsection on ultimate goal vs. proxy metrics
% TODO: also consider short discussion on high-level vs. low-level goals
% (also known as mission/purpose/vision vs goals, or goals vs means
% and discussion on goals vs metrics



\subsection{Ultimate and Proxy Metrics}

TODO: explain possible metrics derivation: -- with specific example of learning programming
1. from the ultimate goal -- but make it precise to make it evaluable (if had all the data we need)
2. proxing 1 to make it measurable in the long term (AB experiment)
3. proxing 2 to make it measurable in the short term (live evaluation)
4. proxing 3 to make it work with offline data (to learn hyperparameters + for holdout evaluation)
5. proxing 4 to make it differentiable function (to guide learning, ie. for updates after new s-t interaction)



\subsection{Perceived Flow}

Instead of using objective (factual, measured) data (such as solving time),
we can ask students to provide explicit ``rating'' (subjective/perceived difficulty),
  e.g. by selecting tags after solving a task,
  such as ``too easy'', ``too difficult'', `just right''
  (or possibly even more specific such as ``boring'', ``weird'', ``fun'').
Advantage: flow looks as a good proxy metric to optimize.
Disadvantages:
  bother students with another questions,
  takes time (but this is negligible comparing to solving programming task),
  inaccurate (noise depending on the mood etc.)



\section{Iterative Improvement}
\label{sec:iterative-improvement}

TODO: explain importance of offline analysis and monitoring;
TODO: the term ``human in the loop'' \cite{stupid-tutoring-systems-intelligent-humans}

TODO: AB experiments

TODO: importance of iterative improvement, rule of the loop;
