\chapter{Design of Game}
\label{chap:design-of-game}

\begin{itemize}
\item terminoglogy: game = task? (or all tasks and missions?)
\item Tasks are the most important part of a system for learning programming,
because nearly all the time in the system students spend solving the tasks.
Therefore they need to be fun, entertaining and calling to be solved.
(TODO: ref to the other requirements mentioned in the prev. chapters)
%(TODO: also mention how we underestimated the importance of the entertaining tasks in the first prototype)
\item requirement for adaptive behavior (for outer loop):
  a large pool of diverse problems
\item standard choice: "robot on a grid" and Blockly blocks
\item theme: spaceship in space, collecting diamonds
\item covered concepts: sequences of commands, repeat, while, if, if-else, simple tests (comparing)
\item motivation: intrinsic (fun challenging game + optimal difficulty)
  + simple external motivation scheme (progress through levels)
\item (? target group: Currently, the system primarily targets at children
  between 8-15 years.)
\item to support flow: clear goal, immediate feedback ("post-explanations"),
  (and increasing difficulty, see next chapter)
\end{itemize}
% TODO?: instructions/explanations

\imgW{robomission-task1}{Example of a task in the RoboMission app.}


\section{Game World}  % Space World
\label{sec:robomission.game-world}

\begin{itemize}
\item original version - robot-in-maze, issues:
  \begin{itemize}
  \item not flexible and did not allow for a lot of diverse easy tasks,
  \item which is crucial for adaptive learning system
  \item unnecessarily long programs for simple ideas (non-elegant programs) (TBA: show comparision)
  \item non-elegant tasks
  \item ugly worlds
  \item not very original and entertaining for kids
  \end{itemize}
\item principle: game world itself must be pleasure to look at and fun to play
with (even without a specific goal) \cite{book-of-lenses}
\item requirements (on the topic/theme/world):
  \begin{itemize}
  \item entertaining for children
  \item allowing to create plenty of diverse (and entertaining) tasks, inluding very simple ones
  \item programs should be not too verbose
  \item not limiting adaptability (e.g. story requring a fixed sequence of tasks would be a problem)
  \item (not limiting another topics later)
  \end{itemize}
\item new game: rocket in space - different theme, mechanics, graphics
\item world elements: grid with colored fields (color clouds) and objects: asteroids and meteoroids, diamonds, wormholes (TODO: explain why), final line, spaceship
\item spaceship actions (left, fly, right, shoot) and sensors (color, position), limited energy
\item movement -- always 1 row forward ("continuous flight ahead") --
  advantages:
  (1) mitigates ubiquitous left-right turning confusion;
  (2) shorter programs (REF example);
  disadvantages:
  (1) this behavior is different than what most users initially expect
  (2) underutilization of fields (only 1 field in each row and only once)
      (leading to long worlds).
\item chosen remedy for the 2nd disadvantage: worm holes
  % (another possibilities: multiple spaceships, long worlds)
\item TODO: shooting
\item explosion if: gets outside the track, hits a stone
\item goal: reach last row (blue line), collect all dimanods
\item fully observable world
\item no chance (randomness would complicate evaluation and interpretation
  of data) (there is still some \emph{surprise} in the game for the players:
  ``Will my program solve the task or not?'')
%\item REF to formal grammar for the space world in the next chapter
\end{itemize}

% TODO: use different spaceworld than in Imlementation chapter.
\imgW[0.4]{spaceworld}{Example of Space World.}


\imgW{prototype-pits}{Lesson from the first prototype: world must contain elements that the program cannot check to enable diverse tasks. As we had a natural sensor to check walls, we needed to add pits that could not be checked by any sensor.}


\section{Programs}
\label{sec:robomission.programs}

\begin{itemize}
\item Given an instace of a spaceworld as described in the previous section
(input), the goal is to create a program for the spaceship such that it reaches
the final row and collects all diamonds on the way.
\item Form of the program: Blockly blocks (+ potentially RoboCode based on
Python, ref to grammar in next chapter; ref to Blockly in prev. chapters)
\item Available blocks: depends on the level;
  actions (...), sensing (...), repeat, while, if, if-else.
\item execute current program any time until correct
\item limits: ``lines'', energy.
  (TODO: explain what they are and why/when they are needed).
\item Why line-limit instead of block-limit: too many blocks (consider while
  loop with a compound condition), so it becomes even difficult to count;
  while the number of lines is always reasonable.
  Bonus: the limit can be directly used for text-based programming as well.
\item TODO: design decision: similar to Python as much as possible: for easy transition
(see the transitional strategies section in chapter 2), allows for AB experiments
comparing directly the influence of the text/block environment as done e.g.
in the field study in \cite{comparing-blocks-text-weintrop2017}.
\end{itemize}


% TODO: Image with program
\imgW{robomission-task2}{Task with asteroids.}


\section{Hints}
\begin{itemize}
\item instructions/explanations
\end{itemize}

\imgW{robomission-mini-instruction}{When a student encounters a new programming concept or programming element, the system displays a short instruction.}
\imgW{robomission-mini-explanation}{The system shows an explanation why the exection was unsuccessful.}
\imgW{prototype-instructions}{Instructions in the first prototype. Nobody was reading them. Most people even did not notice there are any instructions.}


\section{Task Editor}  % game editor?
\label{sec:robomission.task-editor}

A part of the system is the task editor (REF: image)
that enables to easily create new tasks.
(TODO: mention how inconvenient it was in the first prototype, especially the tokens...)
(TODO: mention that the task editor is part of the public site, so anybody can create a task, provide a link and ref to details about task format in the next chapter)


\imgW{task-editor}{Task editor allows to create new tasks.}
\imgW{task-editor-vim}{Task editor allows to to write solutions in Python-based RoboCode and edit space world in Vim mode.}
